\chapter*{Agradecimentos}

Gostaria de dedicar primeiramente este trabalho à memória de minha querida mãe Marinete Albuquerque dos Reis. Agradeço por toda amizade, companheirismo, apoio, paciência e amor incondicional. Tenho a sorte de ser filho de uma mulher guerreira, que nos entregou todo amor e dedicação ao longo de toda a sua vida. Saudades eternas de sua risada e de seu afago nas horas que mais precisei! Amo você demais, onde quer que esteja!

Gostaria de agradecer também à meu pai Descartes, minha irmã Fernanda, meu cunhado Maicon e meu sobrinho Guilherme pelo amor e apoio incondicional. Agradeço por estarmos juntos nos momentos mais difíceis de minha vida, sem vocês isso não seria possível!  

Agradeço ao meu orientador Vanderlei e a minha coorientadora Valéria por todo apoio ao longos destes anos. Todas as melhores palavras que eu pudesse colocar nesses agradecimentos seriam poucas para expressar a gratidão que tenho a vocês. Não menos importante, agradeço ao Jefferson que, além de colaborador, um grande amigo. Somente vocês sabem o quanto lutei para transpor mais este degrau na minha vida. Todas as horas de conversas foram de extrema importância para mim! Agradeço imensamente pela amizade que construímos nestes anos!  

Faço aqui também um agradecimento especial a algumas pessoas que passaram pela minha vida nessa jornada: Léo (meu grande irmão), Larissa (minha grande amiga), Rodrigo Bijani, Mário Martins, Victor Carreira, Florita, Isabella, Matias, Bita e Mário. Estarão sempre em meu coração! Agradeço imensamente tudo o que fizeram por mim!

Agradeço a todos os membros do grupo de Problemas inversos em Geofísica (PINGA)!

Aos amigos da pós-graduação em Geofísica e Astronomia.

Aos membros da banca pela contribuição e revisão do meu trabalho.

Ao CNPq pelo apoio financeiro.

À todos os membros da COGE.

