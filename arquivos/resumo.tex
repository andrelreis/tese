\begin{abstract}

Nesta tese, apresento dois resultados teóricos e as respectivas aplicações da 
camada equivalente no processamento e interpretação de dados magnéticos. 
No primeiro deles, mostro que há uma única camada plana e contínua de dipolos, com uma 
determinada direção de magnetização uniforme, capaz de reproduzir, simultaneamente, as 
três componentes do campo de indução magnética produzido por um conjunto arbitrário de fontes.
Esta propriedade é válida independentemente se a direção de magnetização na camada é igual a 
das fontes ou não.
A partir deste resultado teórico, mostro que é possível usar uma camada plana de dipolos com 
direção de magnetização uniforme e arbitrária para estimar as três componentes do campo de 
indução magnética produzido por um conjunto arbitrário de fontes via inversão linear 
de dados de uma única componente.
Resultados com dados sintéticos produzidos por simulações numéricas e dados reais obtidos 
sobre uma amostra de rocha proveniente da cratera de Vredefort, África do Sul,
mostram a utilidade do método no processamento de dados de microscopia magnética e na 
identificação de regiões com maior concentração de minerais magnéticos.
No segundo desenvolvimento teórico apresentado nesta tese, mostro que a distribuição de 
intensidades de momento magnético sobre uma camada plana e contínua de dipolos é toda 
positiva se a direção de magnetização uniforme na camada é igual àquela das fontes verdadeiras.
Usando esta propriedade de positividade, apresento um método iterativo para estimar a direção 
de magnetização uniforme de um conjunto de fontes 3D a partir da inversão de dados de anomalia de 
campo total. A cada iteração, o método resolve um 
problema inverso linear para estimar uma distribuição de intensidades de momento magnético positiva
e um problema inverso não-linear para estimar a direção de magnetização sobre uma camada plana de dipolos.
Ao final do processo, a direção de magnetização uniforme das fontes equivalentes se aproxima daquela 
das fontes verdadeiras.
Testes com dados produzidos por modelos que simulam diferentes cenários geológicos mostram que o 
método pode ser uma ferramenta poderosa para estimar a direção de magnetização uniforme 
de um conjunto de fontes geológicas. Aplicações a dados de aerolevantamento sobre o complexo de Montes Claros
de Goiás, localizado na província alcalina de Goiás, região central do Brasil, sugerem que estas intrusões 
possuem intensa magnetização remanente, o que está em acordo com um estudo independente conduzido 
previamente na mesma área. 

\end{abstract}