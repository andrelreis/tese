\chapter{Metodologia}
\label{chap:metodologia}


\section{Fundamentação teórica}
\label{sec:fundamentacao}

\begin{equation}
\mathbf{B}(x, y, z) = - \nabla \Gamma(x, y, z)
\label{eq:B-true-generic}
\end{equation}

\begin{equation}
\Gamma(x, y, z) = - \gamma_{m} \iiint\limits_{\upsilon} \mathbf{m}(x', y', z') \cdot \nabla\frac{1}{r'} 
\; d\upsilon'
\label{eq:Gamma-potential}
\end{equation}

\begin{equation}
\frac{1}{r'} = \frac{1}{\left[ (x - x')^{2} + (y - y')^{2} + (z - z')^{2} \right]^{\frac{1}{2}}}
\label{eq:inv-r'}
\end{equation}

\begin{equation}
\mathbf{m}(x', y', z') = \begin{bmatrix}
m_{x}(x', y', z') \\
m_{y}(x', y', z') \\
m_{z}(x', y', z')
\end{bmatrix}
\label{eq:mag-vector-true-generic}
\end{equation}

\begin{equation}
B_{\alpha}(x, y, z) = \gamma_{m} \iiint\limits_{\upsilon} 
\mathbf{m}(x', y', z') \cdot \partial_{\alpha} \nabla \frac{1}{r'} 
\; d\upsilon' \: , \quad \alpha = x, y, z \: ,
\label{eq:B-alpha-true-generic}
\end{equation}
em que $\partial_{\alpha} \equiv \frac{\partial}{\partial \alpha}$ ...

Presumindo que, em qualquer ponto $(x, y, z)$ fora das fontes magnéticas, nenhuma 
componente $B_{\alpha}(x, y, z)$ (equação \ref{eq:B-alpha-true-generic}) varia no tempo, 
o campo de indução magnética $\mathbf{B}(x, y, z)$ (equação \ref{eq:B-true-generic}) é governado
pela lei de Gauss
\begin{equation}
\nabla \cdot \mathbf{B}(x, y, z) = 
\partial_{x} B_{x}(x, y, z) + \partial_{y} B_{y}(x, y, z) + \partial_{z} B_{z}(x, y, z) = 
0
\label{eq:lei-Gauss}
\end{equation}
e a lei de Ampère
\begin{equation}
\nabla \times \mathbf{B}(x, y, z) = \begin{bmatrix}
\partial_{y} B_{z}(x, y, z) - \partial_{z} B_{y}(x, y, z) \\
\partial_{z} B_{x}(x, y, z) - \partial_{x} B_{z}(x, y, z) \\
\partial_{x} B_{y}(x, y, z) - \partial_{y} B_{x}(x, y, z) 
\end{bmatrix} = 
\begin{bmatrix}
0 \\
0 \\
0
\end{bmatrix} \: .
\label{eq:lei-Ampere}
\end{equation}



Considere o vetor unitário 
\begin{equation}
\hat{\mathbf{u}}(I, D) = 
\begin{bmatrix}
\hat{u}_{x}(I, D) \\
\hat{u}_{y}(I, D) \\
\hat{u}_{z}(I, D) 
\end{bmatrix} = 
\begin{bmatrix}
\cos I \, \cos D \\
\cos I \, \sin D \\
\sin I
\end{bmatrix} \: ,
\label{eq:u-hat}
\end{equation}
definido em termos de uma inclinação $I$ e uma declinação $D$.


\begin{equation}
\Delta T(x, y, z) = \hat{\mathbf{u}}(I_{0}, D_{0}) \cdot \mathbf{B}(x, y, z) \: .
\label{eq:Delta-T-true}
\end{equation}


\section{Camada equivalente magnética}
\label{sec:camada-equivalente}


\begin{equation}
\tilde{\mathbf{B}}(x, y, z) = - \nabla \Phi(x, y, z)
\label{eq:B-layer}
\end{equation}

\begin{equation}
\Phi(x, y, z) = - \gamma_{m} \int\limits_{-\infty}^{+\infty}\int\limits_{-\infty}^{+\infty} 
p(x'', y'', z_{c}) \, \hat{\mathbf{u}}(I, D) \cdot \nabla \frac{1}{r''} \; dS'' \: , \quad z_{c} > z \: ,
\label{eq:Phi-potential}
\end{equation}

\begin{equation}
\frac{1}{r''} = \frac{1}{\left[ (x - x'')^{2} + (y - y'')^{2} + (z - z_{c})^{2} \right]^{\frac{1}{2}}}
\label{eq:inv-r''}
\end{equation}

\begin{equation}
\tilde{B}_{\alpha}(x, y, z) = \gamma_{m} \int\limits_{-\infty}^{+\infty}\int\limits_{-\infty}^{+\infty} 
p(x'', y'', z_{c}) \, \hat{\mathbf{u}}(I, D) \cdot \partial_{\alpha} \nabla \frac{1}{r''} \; dS'' 
\: , \quad \alpha = x, y, z \: ,
\label{eq:B-alpha-layer}
\end{equation}

\begin{equation}
\Delta\tilde{T}(x, y, z) = \hat{\mathbf{u}}(I_{0}, D_{0}) \cdot \tilde{\mathbf{B}}(x, y, z) \: .
\label{eq:Delta-T-layer}
\end{equation}


\section{Consequências das Leis de Gauss e Amp{\`e}re}
\label{sec:Gauss-Ampere}

Nesta seção, mostrarei que ...

Considere três camadas equivalentes definidas na mesma coordenada vertical $z = z_{c}$, 
com distribuições de intensidade de momento magnético 
$p(x'', y'', z_{c}) \equiv p_{(\alpha)}$ e direções de magnetização uniforme definidas por 
$\hat{\mathbf{u}}^{(\alpha)} \equiv \hat{\mathbf{u}}(I_{(\alpha)}, D_{(\alpha)})$,
com componentes Cartesianas $\hat{u}^{(\alpha)}_{\beta} \equiv \hat{u}_{\beta}(I_{(\alpha)}, D_{(\alpha)})$ 
(equação \ref{eq:u-hat}), em que $\alpha = x, y, z$, $\beta = x, y, z$. 

Considere também, momentaneamente, que cada camada $\alpha$ produz a componente 
$\tilde{B}_{\alpha}(x, y, z)$ (equação \ref{eq:B-alpha-layer}) de um mesmo 
campo de indução magnética $\tilde{\mathbf{B}}(x, y, z)$ (equação \ref{eq:B-layer}).

Calculando o divergente do campo de indução magnética 
$\tilde{\mathbf{B}}(x, y, z)$ (equação \ref{eq:B-layer}) produzido pela camada equivalente 
e trocando a ordem das derivadas parciais, obtemos 
\begin{equation}
\nabla \cdot \tilde{\mathbf{B}}(x, y, z) = I_{1} + I_{2} + I_{3} \: ,
\label{eq:divergent-layer}
\end{equation}
em que os termos são integrais de superfície dadas por
\begin{equation*}
\begin{split}
I_{1} &= \gamma_{m} \iint
p_{(x)} \hat{u}^{(x)}_{x} \partial_{xxx} \frac{1}{r''} +
p_{(y)} \hat{u}^{(y)}_{x} \partial_{yyx} \frac{1}{r''} +
p_{(z)} \hat{u}^{(z)}_{x} \partial_{zzx} \frac{1}{r''} \; dS'' \\
I_{2} &= \gamma_{m} \iint
p_{(x)} \hat{u}^{(x)}_{y} \partial_{xxy} \frac{1}{r''} +
p_{(y)} \hat{u}^{(y)}_{y} \partial_{yyy} \frac{1}{r''} +
p_{(z)} \hat{u}^{(z)}_{y} \partial_{zzy} \frac{1}{r''} \; dS'' \\
I_{3} &= \gamma_{m} \iint
p_{(x)} \hat{u}^{(x)}_{z} \partial_{xxz} \frac{1}{r''} +
p_{(y)} \hat{u}^{(y)}_{z} \partial_{yyz} \frac{1}{r''} +
p_{(z)} \hat{u}^{(z)}_{z} \partial_{zzz} \frac{1}{r''} \; dS''
\end{split} \:\: .
\end{equation*}
Por conveniência, os limites de integração $-\infty$ e $+\infty$ foram omitidos.
Sabemos que as derivadas primeiras $\partial_{\beta} \frac{1}{r''}$, $\beta = x, y, z$, de 
$\frac{1}{r''}$ (equação \ref{eq:inv-r''}) são funções harmônicas. Esta propriedade permite 
reescrever as integrais $I_{1}$, $I_{2}$ e $I_{3}$ da seguinte forma:
\begin{equation}
\begin{split}
I_{1} &= \gamma_{m} \iint
\left[ p_{(x)} \hat{u}^{(x)}_{x} - p_{(z)} \hat{u}^{(z)}_{x} \right] \partial_{xxx} \frac{1}{r''} +
\left[ p_{(y)} \hat{u}^{(y)}_{x} - p_{(z)} \hat{u}^{(z)}_{x} \right] \partial_{yyx} \frac{1}{r''} +
\; dS'' \\
I_{2} &= \gamma_{m} \iint
\left[ p_{(x)} \hat{u}^{(x)}_{y} - p_{(z)} \hat{u}^{(z)}_{y} \right] \partial_{xxy} \frac{1}{r''} +
\left[ p_{(y)} \hat{u}^{(y)}_{y} - p_{(z)} \hat{u}^{(z)}_{y} \right] \partial_{yyy} \frac{1}{r''} +
\; dS'' \\
I_{3} &= \gamma_{m} \iint
\left[ p_{(x)} \hat{u}^{(x)}_{z} - p_{(z)} \hat{u}^{(z)}_{z} \right] \partial_{xxz} \frac{1}{r''} +
\left[ p_{(y)} \hat{u}^{(y)}_{z} - p_{(z)} \hat{u}^{(z)}_{z} \right] \partial_{yyz} \frac{1}{r''} +
\; dS''
\label{eq:divergent-layer-I123}
\end{split} \:\: .
\end{equation}
Dessa forma, fica evidente que os integrandos de $I_{1}$, $I_{2}$ e $I_{3}$ (equação \ref{eq:divergent-layer-I123}) 
são combinações lineares de funções linearmente independentes, que são derivadas 
terceiras da função $\frac{1}{r''}$ (equação \ref{eq:inv-r''}).
Note que, para o campo de indução magnética $\tilde{\mathbf{B}}(x, y, z)$ (equação \ref{eq:B-layer}) 
respeitar a lei de Gauss (equação \ref{eq:lei-Gauss}), é necessário que as integrais $I_{1}$, $I_{2}$ e $I_{3}$ 
sejam todas nulas.
Neste caso, para que isso aconteça sem que as distribuições de intensidade de momento magnético 
$p_{(x)}$, $p_{(y)}$ e $p_{(z)}$ sejam nulas em todos os pontos $(x'', y'', z_{c})$, é necessário 
que os termos entre colchetes
nos integrandos de $I_{1}$, $I_{2}$ e $I_{3}$ sejam nulos em todos os pontos $(x'', y'', z_{c})$.
Isso, por sua vez, só é possível se as distribuições de intensidade de momento magnético 
e as direções de magnetização uniforme das três camadas forem iguais entre si, isto é, se 
$p_{(x)} = p_{(y)} = p_{(z)}$ e 
$\hat{\mathbf{u}}^{(x)} = \hat{\mathbf{u}}^{(y)} = \hat{\mathbf{u}}^{(z)}$.

Calculando o rotacional do campo de indução magnética 
$\tilde{\mathbf{B}}(x, y, z)$ (equação \ref{eq:B-layer}) produzido pela camada 
e trocando a ordem das derivadas parciais, obtemos
\begin{equation}
\nabla \times \tilde{\mathbf{B}}(x, y, z) = \begin{bmatrix}
\gamma_{m} \iint
\left[ p_{(z)} \, \hat{\mathbf{u}}^{(z)} - p_{(y)} \, \hat{\mathbf{u}}^{(y)} \right] \cdot 
\partial_{yz} \nabla \frac{1}{r''} \; dS'' \\
\gamma_{m} \iint
\left[ p_{(x)} \, \hat{\mathbf{u}}^{(x)} - p_{(z)} \, \hat{\mathbf{u}}^{(z)} \right] \cdot 
\partial_{xz} \nabla \frac{1}{r''} \; dS'' \\
\gamma_{m} \iint
\left[ p_{(y)} \, \hat{\mathbf{u}}^{(y)} - p_{(x)} \, \hat{\mathbf{u}}^{(x)} \right] \cdot 
\partial_{xy} \nabla \frac{1}{r''} \; dS''
\end{bmatrix} \: .
\label{eq:curl-layer}
\end{equation}
Novamente, os limites de integração $-\infty$ e $+\infty$ foram omitidos por conveniência.
Note que os integrandos das componentes do rotacional de $\tilde{\mathbf{B}}(x, y, z)$ 
(equação \ref{eq:curl-layer}) são combinações lineares de funções linearmente 
independentes, que são derivadas terceiras de $\frac{1}{r''}$ (equação \ref{eq:inv-r''}).
Sabemos que, para o campo de indução magnética $\tilde{\mathbf{B}}(x, y, z)$ 
(equação \ref{eq:B-layer}) respeitar a lei de Ampère (equação \ref{eq:lei-Ampere}), é necessário que as 
três componentes de $\nabla \times \tilde{\mathbf{B}}(x, y, z)$ sejam nulas. 
Neste caso, para que isso aconteça sem que as distribuições de intensidade de momento magnético 
$p_{(x)}$, $p_{(y)}$ e $p_{(z)}$ sejam 
nulas em todos os pontos $(x'', y'', z_{c})$, é necessário que os termos entre colchetes
nos integrandos da equação \ref{eq:curl-layer} sejam nulos em todos os pontos $(x'', y'', z_{c})$.
Mais uma vez, isso só é possível se as distribuições de intensidade de momento magnético 
e as direções de magnetização uniforme das três camadas forem iguais entre si, isto é, se 
$p_{(x)} = p_{(y)} = p_{(z)}$ e 
$\hat{\mathbf{u}}^{(x)} = \hat{\mathbf{u}}^{(y)} = \hat{\mathbf{u}}^{(z)}$.

De acordo com estes resultados teóricos, a hipótese de que cada componente $\tilde{B}_{\alpha}(x, y, z)$ 
(equação \ref{eq:B-alpha-layer}), $\alpha = x, y, z$, de um mesmo campo de indução magnética $\tilde{\mathbf{B}}(x, y, z)$ 
(equação \ref{eq:B-layer}) pode ser reproduzida por uma camada equivalente diferente resulta na 
violação das leis de Gauss (equação \ref{eq:lei-Gauss}) de Ampère (equação \ref{eq:lei-Ampere}).
Isto significa que há uma única camada equivalente localizada em $z = z_{c}$, 
com uma determinada distribuição de intensidades de momento magnético $p(x'', y'', z_{c})$ e uma 
determinada direção de magnetização uniforme $\hat{\mathbf{u}}(I, D)$ capaz de reproduzir, simultaneamente,
as três componentes $\tilde{B}_{\alpha}(x, y, z)$ (equação \ref{eq:B-alpha-layer}), $\alpha = x, y, z$, 
de um campo de indução magnética $\tilde{\mathbf{B}}(x, y, z)$ (equação \ref{eq:B-layer}).

Considere agora a existência de uma camada equivalente localizada em $z = z_{c}$, com uma determinada 
distribuição de intensidades de momento magnético $p(x'', y'', z_{c})$ e uma determinada 
direção de magnetização uniforme $\hat{\mathbf{u}}(I, D)$, que produz um campo de indução 
magnética $\tilde{\mathbf{B}}(x, y, z)$ (equação \ref{eq:B-layer}). Adicionalmente, 
considere que uma componente $\tilde{B}_{\alpha}(x, y, z)$ (equação \ref{eq:B-alpha-layer}) 
do campo produzido por esta camada reproduz a componente $B_{\alpha}(x, y, z)$ (equação \ref{eq:B-alpha-true-generic}) 
do campo produzido por uma fonte magnética arbitrária em todos os pontos $(x, y, z)$ localizados fora das fontes e acima 
da camada ($z < z_{c}$). Utilizando o resultado teórico exposto acima, mostra-se que 
esta mesma camada deve obrigatoriamente reproduzir as outras duas componentes do campo produzido 
pelas fontes magnéticas. Caso contrário, cada componente do campo $\tilde{B}_{\alpha}(x, y, z)$ 
(equação \ref{eq:B-alpha-layer}) produzido pela camada, e consequentemente do campo 
$B_{\alpha}(x, y, z)$ (equação \ref{eq:B-alpha-true-generic}) produzido pela fonte arbitrária, 
deverá ser reproduzida por uma camada equivalente diferente, o que violaria as leis de 
Gauss (equação \ref{eq:lei-Gauss}) de Ampère (equação \ref{eq:lei-Ampere}).


\section{Distribuição de momentos positiva}
\label{sec:distribuicao-positiva}
