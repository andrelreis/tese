\chapter{Metodologia}
\label{chap:metodologia}


\section{Fundamentação teórica}
\label{sec:fundamentacao}

\begin{equation}
\mathbf{B}(x, y, z) = - \nabla \Gamma(x, y, z)
\label{eq:B-true-generic}
\end{equation}

\begin{equation}
\Gamma(x, y, z) = - \gamma_{m} \iiint\limits_{\upsilon} \mathbf{m}(x', y', z') \cdot \nabla\frac{1}{r'} 
\; d\upsilon'
\label{eq:Gamma-generic}
\end{equation}

\begin{equation}
\frac{1}{r'} = \frac{1}{\left[ (x - x')^{2} + (y - y')^{2} + (z - z')^{2} \right]^{\frac{1}{2}}}
\label{eq:inv-r'}
\end{equation}

\begin{equation}
\mathbf{m}(x', y', z') = \begin{bmatrix}
m_{x}(x', y', z') \\
m_{y}(x', y', z') \\
m_{z}(x', y', z')
\end{bmatrix}
\label{eq:mag-vector-true-generic}
\end{equation}

\begin{equation}
B_{\alpha}(x, y, z) = \gamma_{m} \iiint\limits_{\upsilon} 
\mathbf{m}(x', y', z') \cdot \partial_{\alpha} \nabla \frac{1}{r'} 
\; d\upsilon' \: , \quad \alpha = x, y, z \: ,
\label{eq:B-alpha-true-generic}
\end{equation}
em que $\partial_{\alpha} \equiv \frac{\partial}{\partial \alpha}$ ...

Presumindo que, em qualquer ponto $(x, y, z)$ fora das fontes magnéticas, nenhuma 
componente $B_{\alpha}(x, y, z)$ (equação \ref{eq:B-alpha-true-generic}) varia no tempo, 
o campo de indução magnética $\mathbf{B}(x, y, z)$ (equação \ref{eq:B-true-generic}) é governado
pela lei de Gauss
\begin{equation}
\nabla \cdot \mathbf{B}(x, y, z) = 
\partial_{x} B_{x}(x, y, z) + \partial_{y} B_{y}(x, y, z) + \partial_{z} B_{z}(x, y, z) = 
0
\label{eq:lei-Gauss}
\end{equation}
e a lei de Ampère
\begin{equation}
\nabla \times \mathbf{B}(x, y, z) = \begin{bmatrix}
\partial_{y} B_{z}(x, y, z) - \partial_{z} B_{y}(x, y, z) \\
\partial_{z} B_{x}(x, y, z) - \partial_{x} B_{z}(x, y, z) \\
\partial_{x} B_{y}(x, y, z) - \partial_{y} B_{x}(x, y, z) 
\end{bmatrix} = 
\begin{bmatrix}
0 \\
0 \\
0
\end{bmatrix} \: .
\label{eq:lei-Ampere}
\end{equation}



Considere o vetor unitário 
\begin{equation}
\hat{\mathbf{u}}(I, D) = 
\begin{bmatrix}
\hat{u}_{x}(I, D) \\
\hat{u}_{y}(I, D) \\
\hat{u}_{z}(I, D) 
\end{bmatrix} = 
\begin{bmatrix}
\cos I \, \cos D \\
\cos I \, \sin D \\
\sin I
\end{bmatrix} \: ,
\label{eq:u-hat}
\end{equation}
definido em termos de uma inclinação $I$ e uma declinação $D$.


\begin{equation}
\Delta T(x, y, z) = \hat{\mathbf{u}}(I_{0}, D_{0}) \cdot \mathbf{B}(x, y, z) \: .
\label{eq:Delta-T-true}
\end{equation}


\section{Camada equivalente magnética}
\label{sec:camada-equivalente}


\begin{equation}
\tilde{\mathbf{B}}(x, y, z) = - \nabla \Phi(x, y, z)
\label{eq:B-layer}
\end{equation}

\begin{equation}
\Phi(x, y, z) = - \gamma_{m} \int\limits_{-\infty}^{+\infty}\int\limits_{-\infty}^{+\infty} 
p(x", y", z_{c}) \hat{\mathbf{u}}(I, D) \cdot \nabla \frac{1}{r"} \; dS" \: , \quad z_{c} > z \: ,
\label{eq:Phi-generic}
\end{equation}

\begin{equation}
\frac{1}{r"} = \frac{1}{}
\end{equation}


\section{Consequências das Leis de Gauss e Amp{\`e}re}
\label{sec:Gauss-Ampere}




\section{Distribuição de momentos positiva}
\label{sec:distribuicao-positiva}
