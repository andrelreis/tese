\chapter{Introdução}

Magnetic measurements of a single component of the magnetic field contain information about the other components. For this reason, maps of the x- and y-components of the magnetic field can be estimated from z-component measurements. Vector field maps are valuable tool for magnetic interpretation of samples with high spatial variability of magnetization. These maps can provide a comprehensive information about the spatial distribution of magnetic carriers. Moreover, it can be useful for characterizing isolated areas over the samples or investigating the spatial magnetization distribution of bulk samples, in submillimeter and millimeter scales. The amplitude of total-field vector map calculated from three estimated components can also show regions devoid of magnetic sources. Several techniques in Fourier domain do not require the formulation of an inverse problem for estimating the components of the magnetic field \citep[e.g.,][]{lourenco_morrison_1973,lima_weiss_2009}. However, it is possible to estimate the magnetic field vector in spatial domain approach based on equivalent-layer technique \citep[e.g.,][]{li_li_2014}

We illustrate in this section how equivalent-layer technique can be used to estimate the three components of the magnetic field using magnetic microscopy data. We applied this method in a metamorphic rock from Vredefort Dome in South Africa. The Vredefort Dome comprises by $90$-km central uplift of a $300$-km eroded impact structure \citep{lana_etal_2003}. Once known the magnetization direction of the geological sample, we estimate a magnetic-moment distribution over the equivalent layer formulating an inverse problem and then calculate the three components of the magnetic field.  
  

