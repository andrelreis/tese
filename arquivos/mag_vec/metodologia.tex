\chapter{Metodologia}
\label{chap:metodolodia_part2}

\section{Fundamentação teórica para o cálculo das componentes do vetor magnético}

Com o intuito de investigarmos que o cálculo das componentes do vetor magnético não depende da direção de magnetização, iremos explicar teoricamente como este processo é possível no contexto da camada equivalente. Em situações práticas, somente uma das componentes é medida no laboratório a uma distância fixa da superfície da amostra. Considerando que as medições são realizadas em regiões livres de fontes e, portanto, externas as amostras de rocha. Além disso, consideramos que o cálculo das componentes do vetor magnético não depende do tipo de fonte, bem como de sua configuração espacial. Assumimos também que o campo magnético não varia com o tempo ou que esta variação seja tão pequena que pode ser desprezada ao longo das medições. Consequentemente, o campo de indução magnética $\mathbf{B}(x,y,z)$ é governado pela lei de Gauss

\begin{equation}
\nabla . \mathbf{B} (x,y,z) = 0
\label{eq:gauss_B}
\end{equation}
e pela lei de Ampère 

\begin{equation}
\nabla \times \mathbf{B} (x,y,z) = 0.
\label{eq:ampere_B}
\end{equation}
Portanto, em coordenadas Cartesianas, a equação \ref{eq:ampere_B} corresponde a 

\begin{equation}
\begin{split}
\partial_y B_z - \partial_z B_y = 0 \\
\partial_z B_x - \partial_x B_z = 0 \\
\partial_x B_y - \partial_y B_x = 0, 
\end{split}
\label{eq:ampere_B_condicao}
\end{equation}
em que $\partial_\alpha \equiv \dfrac{\partial}{\partial \alpha}$ é denotado como a derivada parcial em relação a coordenada $\alpha$, $\alpha=x,y,z$. 

Considere que $\tilde{\mathbf{B}}(x,y,z)$ é o campo de indução magnética produzido por uma camada contínua de dipolos que tem direção de magnetização constante $\hat{\mathbf{m}} (\mathbf{q})$, analogamente a equação \ref{eq:mag_vec}, posicionada a uma profundidade $z=z_c$ abaixo do plano de observação. O campo de indução magnética produzido por esta camada é dado por 

\begin{equation}
\tilde{\mathbf{B}}(x, y, z) = \gamma_{m} \, \tilde{\mathbf{M}}(x, y, z) \: \hat{\mathbf{m}}(\mathbf{q}) \: ,
\label{eq:B-eqlayer}
\end{equation}
em que $\tilde{\mathbf{M}}(x, y, z)$ é uma matriz dada por 
\begin{equation}
\tilde{\mathbf{M}}(x, y, z) = \begin{bmatrix}
\partial_{xx} \Phi(x, y, z) & 
\partial_{xy} \Phi(x, y, z) &
\partial_{xz} \Phi(x, y, z) \\
\partial_{xy} \Phi(x, y, z) & 
\partial_{yy} \Phi(x, y, z) &
\partial_{yz} \Phi(x, y, z) \\
\partial_{xz} \Phi(x, y, z) & 
\partial_{yz} \Phi(x, y, z) &
\partial_{zz} \Phi(x, y, z)
\end{bmatrix} \quad ,
\label{eq:M-matriz-eqlayer-B}
\end{equation}
com elementos 
$\partial_{\alpha\beta} \Phi(x, y, z) \equiv 
\frac{\partial^{2} \Phi(x, y, z)}{\partial \alpha \partial \beta}$, 
$\alpha, \beta = x, y, z$, representando as derivadas da função hamônica 
\begin{equation}
\Phi(x, y, z) = \int\limits_{-\infty}^{+\infty}\int\limits_{-\infty}^{+\infty}
\frac{p(x'', y'', z_{c}) \: dS''}
{\left[ (x-x'')^2 + (y-y'')^2 + (z-z_{c})^2 \right]^{\frac{1}{2}}} \: ,
\quad z_{c} > z \: .
\label{eq:Phi-integral-superficie-B}
\end{equation}
Nesta equação $x''$, $y''$ and $z_{c}$ são as coordenadas do elemento de área $dS''$, que tem momento magnético por unidade de área definido pela função $p(x'', y'', z_{c})$. De maneira simplificada, as componentes do vetor magnético podem ser reescritas como

\begin{equation}
\tilde{B}_{\alpha}(x, y, z) = \gamma_{m} \, \partial_{\alpha\beta} \Phi(x, y, z) m_{\beta} \: ,
\label{eq:B-eqlayer-notacao-einstein}
\end{equation}
em que $\tilde{B}_{\alpha} (x,y,z)$ é a componente $\alpha$, $\alpha=x,y,z$, do campo de indução magnética e $m_{\beta}$ é a componente cartesiana $\beta$, $\beta =x,y,z$, da magnetização da camada. A componente vertical do campo de indução magnética é dada, por exemplo, por 

\begin{equation}
\tilde{B}_{z}(x, y, z) = \gamma_{m} \,[ \partial_{xz} \Phi^{\dagger}(x, y, z) m_{x}^{\dagger} + \partial_{yz} \Phi^{\dagger}(x, y, z) m_{y}^{\dagger} + \partial_{zz} \Phi^{\dagger}(x, y, z) m_{z}^{\dagger} ]  \: ,
\label{eq:Bz-eqlayer}
\end{equation}
em que a função harmônica $\partial_{xz} \Phi^{\dagger}(x, y, z)$ é dada por 

\begin{equation}
\Phi^{\dagger}(x, y, z) = \int\limits_{-\infty}^{+\infty}\int\limits_{-\infty}^{+\infty}
\frac{p^{\dagger}(x'', y'', z_{c}) \: dS''}
{\left[ (x-x'')^2 + (y-y'')^2 + (z-z_{c})^2 \right]^{\frac{1}{2}}} \: .
\label{eq:Phi-integral-superficie-Bz}
\end{equation}
A função $p^{\dagger}(x'', y'', z_{c})$ descreve a distribuição de momentos magnéticos por unidade de área relativa a componente $\tilde{B}_{z}(x, y, z)$ do campo de indução magnética gerado por uma camada contínua e direção de magnetização $\hat{\mathbf{m}}^{\dagger}(\mathbb{q})$. Analogamente, podemos reescrever as outras duas componentes do campo como

\begin{equation}
\tilde{B}_{x}(x, y, z) = \gamma_{m} \, [\partial_{xx} \Phi^{\ddagger}(x, y, z) m_{x}^{\ddagger} + \partial_{xy} \Phi^{\ddagger}(x, y, z) m_{y}^{\ddagger} + \partial_{xz} \Phi^{\ddagger}(x, y, z) m_{z}^{\ddagger}]   \: 
\label{eq:Bx-eqlayer}
\end{equation}
e 

\begin{equation}
\tilde{B}_{y}(x, y, z) = \gamma_{m} \, [\partial_{xy} \Phi^{\sharp}(x, y, z) m_{x}^{\sharp} + \partial_{yy} \Phi^{\sharp}(x, y, z) m_{y}^{\sharp} + \partial_{yz} \Phi^{\sharp}(x, y, z) m_{z}^{\sharp}]   \: .
\label{eq:By-eqlayer}
\end{equation}
Note que as equações \ref{eq:Bz-eqlayer}, \ref{eq:Bx-eqlayer} e \ref{eq:By-eqlayer} representam as camadas equivalentes que geram as componentes do vetor magnético com suas respectivas direções magnetização e distribuições de momentos magnéticos. 



\section{Problema direto para a camada equivalente da componente vertical do campo}
%
%Consider a right-handed Cartesian coordinate system with $z$ being oriented positively downward, $x$ directed to the north, and $y$ directed toward to the east. Let $\mathbf{B}_z^o$ be the $N \times 1$ vector whose the $i$th element $B_z^i$ is the z-component of the induction magnetic field at the observation point $(x^i,y^i,z^i)$ over a pĺane above a magnetic sample. In order to investigate the magnetic field components, we retrieve the magnetic data produced by rock sampĺe using a layer composed by $M$ dipoles with unit volume, all of them positioned at a constant depth of $z=h$. Mathematically, the predicted z-component of magnetic field produced by the set of dipoles at the point $(x^i,y^i,z^i)$ is given by
%
%\begin{equation}
%B_{z}^{i}  = \sum_{j=1}^{M} m^j a_z^{ij},
%\label{eq:pred_data_ith}
%\end{equation}
%where $m^j$ is the magnetic moment of the $j$th prism and 
%
%\begin{equation}
%a_{z}^{ij}  = \gamma_m \mathbf{M}_z^{ij} \hat{\mathbf{m}},
%\label{eq:az_ij}
%\end{equation}
%in which $\gamma_m$ is a constant proportional to the vacuum permeability, $\mathbf{M}_z^{ij}$ is a $3 \times 1$ vector equal to 
%
%\begin{equation}
%\mathbf{M}_z^{ij}  =  [\partial_{xz} \phi^{ij} \, \partial_{yz} \phi^{ij} \, \partial_{zz} \phi^{ij}]^T,
%\label{eq:Mz_ij}
%\end{equation}
%where $\partial_{\lambda z} \phi^{ij}$, $\lambda=x,y,z$, is the second derivative with respect to the Cartesian coordinates $x^i$, $y^i$ and $z^i$ of the observation points of the scalar function $\phi^{ij}$ given by
%
%\begin{equation}
%\phi^{ij} = \dfrac{1}{\sqrt{(x^i - x^j)^2 + (y^i- y^j )^2 + (z^i - h)^2 } }
%\label{eq:phi_ij}
%\end{equation}
%in which $x^j$, $y^j$ and $h$ are the Cartesian coordinates of the $j$th dipole composing the layer. The $\hat{\mathbf{m}}$ is a $3 \times 1$ unit vector with the magnetization direction of all equivalent sources equal to
%
%\begin{equation}
%\hat{\mathbf{m}} =
%\left[ \begin{array}{c}
%\cos I \cos D \\
%\cos I \sin D \\
%\sin I     
%\end{array} \right] ,
%\label{eq:main_field}
%\end{equation}
%where the $I$ and $D$ are the inclination and declination, respectively. This modelling is solved by using a Python library called Fatiando a Terra \citep{uieda2013}. In matrix notation, the predicted z-component of magnetic field produced by the model is given by
%
%\begin{equation}
%\mathbf{B}_{z}^{p}  = \mathbf{A_z} \mathbf{m}
%\label{eq:pred_vec}
%\end{equation}
%in which $\mathbf{B}_{z}^{p}$ is an $N$-dimensional vector whose $i$th element is the predicted z-component of magnetic field at the point $(x^i,y^i,z^i)$, $\mathbf{A_z} $ is an $N \times M$ sensitivity matrix whose $ij$th element is defined by the harmonic function $a_z^{ij}$ (equation \ref*{eq:az_ij}), and $\mathbf{m}$ is the $M$-dimensional parameter vector whose $j$th element is the magnetic moment of the $j$th positioned at the point $(x^j,y^j,h)$. Moreover, the parameter vector $\mathbf{m}$ represents the magnetic-moment distribution over the layer.

\section{Problema inverso e o cálculo das componentes do vetor magnético}
%The inverse problem consists in estimating the magnetic-moment distribution $\mathbf{m}$ by solving a linear system for minimizing the difference between observed data $\mathbf{B}_z^o$ and the predicted data $\mathbf{B}_{z}^{p}$ (equation \ref*{eq:pred_vec}). In order to estimate a stable solution $\mathbf{m}^\ast$, we solve the constrained problem of minimizing the goal function
%
%\begin{equation}
%\Gamma (\mathbf{m}) = \parallel \mathbf{B}_{z}^{o} - \mathbf{B}_{z}^{p}(\mathbf{m}) \parallel_2^2 + \mu \parallel \mathbf{m} \parallel_2^2
%\label{eq:goal_func}
%\end{equation}
%where the first and the second term of equation \ref*{eq:goal_func} are the data-misfit function and the zeroth-order Tikhonov regularization function, $\mu$ is the regularizing parameter and $\parallel . \parallel_2^2$ denotes the Euclidian norm. The least-squares estimate of the parameter vector $\mathbf{m}^\ast$ is given by 
%
%\begin{equation}
%	\mathbf{m}^\ast = \left( \mathbf{A}_z^T \mathbf{A}_z + \mu \mathbf{I} \right)^{-1} \mathbf{A}_z^T \mathbf{B}_z^o
%\label{eq:ls_estimator}
%\end{equation}
%in which the superscript $T$ stands for a transpose and $\mathbf{I}$ is an indentity matrix of order M. After estimating a magnetic-moment distribution $\mathbf{m}^\ast$, we can calculate the two other components of the magnetic field using the relations 
%
%\begin{equation}
%\mathbf{B}_{x}^{p}  = \mathbf{A_x} \mathbf{m}^\ast
%\label{eq:pred_vec_x}
%\end{equation}
%and 
%
%\begin{equation}
%\mathbf{B}_{y}^{p}  = \mathbf{A_y} \mathbf{m}^\ast
%\label{eq:pred_vec_y}
%\end{equation}
%in which  $\mathbf{B}_{x}^{p}$ and $\mathbf{B}_{y}^{p}$ are, respectively, the N-dimensional predicted vectors of x- and y-components of the magnetic field and  $\mathbf{A}_{x}^{p}$ and $\mathbf{A}_{y}^{p}$ are $N \times M$ matrices whose $ij$th elements are, respectively, given by
%
%\begin{equation}
%a_{x}^{ij}  = \gamma_m \mathbf{M}_x^{ij} \hat{\mathbf{m}},
%\label{eq:ax_ij}
%\end{equation}
%and 
%\begin{equation}
%a_{y}^{ij}  = \gamma_m \mathbf{M}_y^{ij} \hat{\mathbf{m}},
%\label{eq:ay_ij}
%\end{equation}
%where
%
%\begin{equation}
%\mathbf{M}_x^{ij}  =  [\partial_{xx} \phi^{ij} \, \partial_{xy} \phi^{ij} \, \partial_{xz} \phi^{ij}]^T,
%\label{eq:Mx_ij}
%\end{equation}
%and
%\begin{equation}
%\mathbf{M}_y^{ij}  =  [\partial_{yx} \phi^{ij} \, \partial_{yy} \phi^{ij} \, \partial_{yz} \phi^{ij}]^T,
%\label{eq:My_ij}
%\end{equation}
%in which $\partial_{\lambda \sigma} \phi^{ij}$, $\lambda = x,y$ and $\sigma = x,y,z$ are the second derivatives of the scalar function $\phi^{ij}$ with respect to the Cartesian coordinates $x^i$, $y^i$ and $z^i$ of the observation points, analogously to equation \ref*{eq:az_ij}. Furthermore, We can also calculate the amplitude of total field by applying the following equation 
%
%\begin{equation}
%\mathbf{B} = \sqrt{ \mathbf{B}_{x}^{p^2} + \mathbf{B}_{y}^{p^2} + \mathbf{B}_{z}^{p^2} }  
%\label{eq:total_field}
%\end{equation}
%where  $\mathbf{B}_{x}^{p}$, $\mathbf{B}_{y}^{p} $ and  $\mathbf{B}_{z}^{p}$ are the x-,y- and z-components of the magnetic field and $\mathbf{B}$ is the amplitude of total magnetic field vector. 



  


 

