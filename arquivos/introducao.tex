\chapter{Introdução}
\label{chap:introducao}

A maioria dos métodos magnéticos requer o conhecimento da direção de magnetização, caso contrário produzem 
informações insatisfatórias sobre as fontes. Este fato tem impulsionado o desenvolvimento de diversas técnicas 
para estimar a direção de magnetização ao longo dos últimos 50 anos. As estratégias para estimar esta quantidade 
podem ser dividida em dois grupos principais. O primeiro grupo compreende aqueles métodos que presumem informações 
prévias acerca da geometria das fontes. O método iterativo apresentado por \cite{bhattacharyya1966} presume 
que as fontes magnéticas tem formato de prismas retangulares. \cite{emilia_massey_1974} aproxima um monte 
submarino por um conjunto de prismas justapostos com direção de magnetização uniforme e intensidade variável. 
\cite{parker_etal_1987} aproximam a geometria de um monte submarino utilizando uma cobertura de faces 
triangulares e estimam a magnetização interna próxima a uma solução uniforme. \cite{parker_etal_1987} apresentaram 
um método que estima a direção de magnetização total e a orientação espacial de uma fonte isolada que possui 
três planos ortogonais de simetria. \cite{kubota2005} aproximaram também um monte submarino por um conjunto 
de prismas justapostos, mas estimaram a direção de magnetização para cada um dos prismas. 
Finalmente, \cite{oliveirajr_etal_2015} aproximam as fontes magnéticas por corpos esféricos de centros conhecidos 
e estimam suas direções de magnetização. O segundo grupo é formado pelos métodos que não presumem informações 
sobre a geometria das fontes. \cite{fedi_etal_1994}, por exemplo, propôs um método que determina a melhor 
direção de magnetização dentre um conjunto de tentativas usadas para realizar sucessivas reduções ao polo no 
domínio de Fourier. \cite{phillips2005} utilizaram integrais de Helbig para estimar as componentes do vetor 
de momento magnético. \cite{tontini_pedersen_2008} estenderam o método de Phillips utilizando as mesmas 
integrais de Helbig para estimar a direção de magnetização e sua magnitude, e fornecendo informações sobre a 
posição do centro de distribuição de magnetização. \cite{lelievre_oldenburg_2009} desenvolveram um método para 
estimar a direção de magnetização em cenários geológicos complexos. Este método aproxima a subsuperfície por um 
grid de prismas justapostos e estima as componentes do vetor magnetização para cada prisma. 
Além destes métodos, existem aqueles que são baseados na correlação de quantidades potenciais 
\citep[e.g.,][]{dannemiller_li_2006,gerovska_etal_2009,liu_etal_2015,zhang_etal_2018}. 

Estimar a direção de magnetização é extremamente importante não só para interpretação, mas também para o 
processamento de anomalia de campo total. Uma técnica no domínio do espaço comumente utilizada para esta 
finalidade é a camada equivalente. Esta técnica foi introduzida na geofísica de exploração por \cite{dampney1969} 
e \cite{emilia_massey_1974} para o processamento de dados gravimétricos e magnéticos, respectivamente. 
Após estes trabalhos pioneiros, esta técnica tem sido utilizada para realizar interpolação 
\citep{cordell_1992, mendonca-silva_1994, barnes-lumley_2011, siqueira_etal_2017}, continuação para cima 
(ou para baixo) \citep{hansen_miyazaki1984, li_oldenburg_2010}, redução ao polo 
\citep{silva_1986, leao-silva_1989, guspi-novara_2009, oliveirajr-etal_2013}, calcular a amplitude do 
campo anômalo \citep{li_li_2014} e para filtrar ruídos de dados de gradiometria \citep{martinez_li_2016}. 
A técnica da camada equivalente consiste em aproximar um conjunto de dados observados por dados produzidos 
por uma camada composta de fontes discretas (e.g., prismas, dipolos ou pontos de massa), que são comumente 
conhecidas como fontes equivalentes. Os dados produzidos por esta camada fictícia (a camada equivalente) 
são chamados de dados preditos. 

Em microscopia magnética, a técnica da camada equivalente é geralmente utilizada para a interpretação da 
distribuição de momentos magnéticos em uma lâmina de rocha. Note que, neste caso, a camada equivalente se 
assemelha a fonte verdadeira (a lâmina de rocha). \cite{weiss2007} apresentou um dos primeiros trabalhos 
utilizando a camada equivalente em microscopia magnética. Os autores apontaram que a distribuição de magnetização 
é inteiramente positiva se a direção de magnetização das fontes equivalentes é igual a direção utilizada para 
uma amostra de rocha magnetizada artificialmente. \cite{baratchart2013} mostraram matematicamente que, 
assumindo uma direção de magnetização uniforme, o problema inverso para estimar a distribuição de momentos 
magnéticos é único. \cite{lima2013} propôs um método no domínio da frequência para investigar soluções tendo 
a direção de magnetização uniforme iguais a das lâminas de rocha. Eles mostraram empiricamente que, neste caso, 
a distribuição de momentos magnéticos na camada é inteiramente positiva.  Em geofísica de exploração, a 
camada equivalente é predominantemente utilizada para o processamento de dados potenciais. Neste sentido, não 
existe relação entre a distribuição de propriedade física estimada sobre a camada equivalente e as fontes 
geológicas verdadeiras. Poucos autores tem direcionado o uso da camada equivalente para a interpretação de fontes 
geológicas. \cite{pedersen1991}, por exemplo, discutiu a relaçao entre o campo potencial e a fonte equivalente. 
\cite{medeiros_silva1996} e \cite{silva-etal2010} estimaram um mapa de magnetização aparente sobre a camada 
utilizando regularizações de Tikhonov e entrópica, respectivamente. \cite{siqueira_etal_2017} estabeleceu a 
relação entre o excesso de massa estimado sobre a camada e o verdadeiro. \cite{li_etal_2014} provaram, 
utilizando uma abordagem no domínio de Fourier, a existência de uma distribuição de momentos magnéticos positiva 
sobre a camada e utilizaram esta propriedade para contornar o problema de instabilidade em baixas latitudes. 
No entanto, estes autores consideraram somente um caso particular no qual as fontes magnéticas têm 
magnetização puramente induzida. 

Nesta tese, apresentamos dois desenvolvimentos teóricos e suas aplicações na interpretação de dados magnéticos 
usando a técnica da camada equivalente. 
O primeiro desenvolvimento teórico é fundamentado nas leis de Gauss para campo magnéticos e de Ampère. 
Provamos matematicamente que, dada uma direção de magnetização uniforme, existe uma única distribuição de 
intensidade de momentos magnéticos na camada equivalente, em uma determinada profundidade constante abaixo 
das observações, que é capaz de reproduzir, simultaneamente, as três componentes de um mesmo campo de indução 
magnética produzido por fontes magnéticas arbitrárias. Esta camada equivalente não requer o conhecimento 
da direção de magnetização verdadeira das fontes magnéticas. Ao contrário, a direção de magnetização da 
distribuição de intensidade de momentos magnéticos na camada equivalente é arbitrária, mas ela deve ser uniforme. 
Testes com dados sintéticos simulando amostras de rochas magnetizadas mostram que a partir de uma das 
componentes do campo de indução magnética conseguimos estimar as outras duas componentes deste campo sem o 
conhecimento da direção de magnetização verdadeira da amostra de rocha simulada. Adicionalmente, calculamos 
amplitude do campo de indução magnética, que é definida como a raiz quadrada da soma dos quadrados das 
componentes horizontais e vertical do campo magnético. Medidas da componente vertical do campo de indução 
magnética realizadas por um microscópio magnético de efeito Hall sobre uma amostra de rocha da cratera de 
Vredefort, Africa do Sul, possibilitaram a determinação das componentes horizontais e da amplitude do campo 
de indução magnética \citep{araujo_etal2019_materials}. 
Estes resultados com dados de microscopia permitiram não só visualizar a estimativa da distribuição de 
magnetização sobre a amostra de rocha da cratera de Vredefort, como também identificar regiões com maior 
concentração de portadores magnéticos.

No segundo desenvolvimento teórico desta tese, provamos matematicamente que a distribuição de momentos 
magnéticos positiva sobre a camada equivalente existe mesmo que a magnetização das fontes verdadeiras seja 
remanente. Esta distribuição positiva de momentos magnéticos ocorre se e somente se 
a direção de magnetização das fontes equivalentes for a mesma das fontes verdadeiras. Esta propriedade 
de positividade da distribuição de momentos ocorre independentemente se a magnetização das fontes verdadeiras 
é puramente induzida ou não.
Amparado nesta propriedade de positividade, apresentamos um método iterativo vinculado que usa a técnica da 
camada equivalente para estimar a direção de magnetização uniforme de fontes arbitrárias invertendo dados de 
anomalia de campo total. Nosso método não presume qualquer informação sobre a geometria das fontes. 
A cada iteração, nosso método resolve (1) um problema linear para estimar uma distribuição de momentos magnéticos 
positiva sobre uma camada de dipolos e (2) um problema inverso não-linear para estimar a direção de magnetização 
uniforme das fontes equivalentes. Testes com dados sintéticos gerados por cenários geológicos diferentes 
mostram que a direção de magnetização estimada converge para a direção verdadeira das fontes verdadeiras. 
Aplicamos também nosso método a dados de campo provenientes da província alcalina de Goiás, sobre o complexo 
de Montes Claros, na região central do Brasil. Nosso resultado está de acordo com os resultados obtidos por 
\citep{zhang_etal_2018} para a mesma área, sugerindo a presença de magnetização remanente e mostrando a boa 
performance do nosso método na interpretação de cenários geológicos complexos. 
