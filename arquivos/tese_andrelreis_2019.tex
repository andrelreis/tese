% Para definir o tipo de documento, descomente apenas
% uma das linhas "\documentclass" abaixo

% comentar uma linha significa colocar "%"
% descomentar uma linha significa remover o "%"

%\documentclass[msc]{on}     % dissertação de mestrado
\documentclass[dsc]{on}     % tese de doutorado
%\documentclass[dscexam]{on} % exame de qualificação
%\documentclass[reportd]{on} % relatório feito durante o doutorado
%\documentclass[reportm]{on} % relatório feito durante o mestrado

% pacotes utilizados
\usepackage[utf8]{inputenc}
\usepackage{amsmath,amssymb}
\usepackage{hyperref}
\usepackage{enumerate} %para gerar listas numeradas
\usepackage{graphicx}  %para figuras eps
\usepackage{subfigure} %para figuras múltimas, com (a), (b), (c), etc.
\usepackage[english, ruled]{algorithm2e}
\usepackage{pdfpages}

% os dois comandos estão com problema e deverão ser
% corrigidos futuramente
%\makelosymbols
%\makeloabbreviations

\begin{document}
  	
  % Título em português
  \title{Inversão Magnética em diferentes escalas}
  
  % Título em inglês
  \foreigntitle{Magnetic Inversion at different scales}
  
  % Autor
  \author{André Luis}{Albuquerque dos Reis}

  % Orientador(a)
  \advisor{Dr.}{Vanderlei}{Coelho de Oliveira Junior}
  %\advisor{Dr.}{Nome do orientador}{Sobrenome}

  % Co-orientadores (pode ser mais de um)
  \coadvisor{Dra.}{Valéria Cristina}{Ferreira Barbosa}
  %\coadvisor{Dr.}{Nome do Co-orientador}{Sobrenome}

  % Examinadores (caso seja um relatório, não modifique as linhas "\examiner")
  %\examiner{Dra.}{Nome da Examinadora}{Sobrenome}
  %\examiner{Dr.}{Nome do Examinador}{Sobrenome}
  %\examiner{Dra.}{Nome da Examinadora}{Sobrenome}
  %\examiner{Dr.}{Nome do Examinador}{Sobrenome}
  %\examiner{Dra.}{Nome da Examinadora}{Sobrenome}

  % Programa de Pós-Graduação 
  \program{GEO}
    
  % Data (mês e ano)
  \date{09}{2019}

  % Palavras-chave
  \keyword{Primeira palavra-chave}
  \keyword{Segunda palavra-chave}
  \keyword{Terceira palavra-chave}

  \maketitle

  %\begin{abstract}

Nesta tese, apresento dois resultados teóricos e as respectivas aplicações da 
camada equivalente no processamento e interpretação de dados magnéticos. 
No primeiro deles, mostro que há uma única camada plana e contínua de dipolos, com uma 
determinada direção de magnetização uniforme, capaz de reproduzir, simultaneamente, as 
três componentes do campo de indução magnética produzido por um conjunto arbitrário de fontes.
Esta propriedade é válida independentemente se a direção de magnetização na camada é igual a 
das fontes ou não.
A partir deste resultado teórico, mostro que é possível usar uma camada plana de dipolos com 
direção de magnetização uniforme e arbitrária para estimar as três componentes do campo de 
indução magnética produzido por um conjunto arbitrário de fontes via inversão linear 
de dados de uma única componente.
Resultados com dados sintéticos produzidos por simulações numéricas e dados reais obtidos 
sobre uma amostra de rocha proveniente da cratera de Vredefort, África do Sul,
mostram a utilidade do método no processamento de dados de microscopia magnética e na 
identificação de regiões com maior concentração de minerais magnéticos.
No segundo desenvolvimento teórico apresentado nesta tese, mostro que a distribuição de 
intensidades de momento magnético sobre uma camada plana e contínua de dipolos é toda 
positiva se a direção de magnetização uniforme na camada é igual àquela das fontes verdadeiras.
Usando esta propriedade de positividade, apresento um método iterativo para estimar a direção 
de magnetização uniforme de um conjunto de fontes 3D a partir da inversão de dados de anomalia de 
campo total. A cada iteração, o método resolve um 
problema inverso linear para estimar uma distribuição de intensidades de momento magnético positiva
e um problema inverso não-linear para estimar a direção de magnetização sobre uma camada plana de dipolos.
Ao final do processo, a direção de magnetização uniforme das fontes equivalentes se aproxima daquela 
das fontes verdadeiras.
Testes com dados produzidos por modelos que simulam diferentes cenários geológicos mostram que o 
método pode ser uma ferramenta poderosa para estimar a direção de magnetização uniforme 
de um conjunto de fontes geológicas. Aplicações a dados de aerolevantamento sobre o complexo de Montes Claros
de Goiás, localizado na província alcalina de Goiás, região central do Brasil, sugerem que estas intrusões 
possuem intensa magnetização remanente, o que está em acordo com um estudo independente conduzido 
previamente na mesma área. 

\end{abstract}   % resumo em português - deve ser utilizado por todos os tipos de documento
  %\begin{foreignabstract}

In this thesis, I present two theoretical results and the applications of the 
equivalent layer for processing and interpreting magnetic data.
In the first one, I show that there is a unique planar and continuous layer of dipoles,
with a given uniform magnetization direction, that is able to reproduce, simultaneously,
the three components of the magnetic induction field produced by an arbitrary set of 
sources. This property holds true regardless of whether the magnetization direction of 
the layer is equal to the that of the sources or not.
From this theoretical result, I show that it is possible to use a planar layer of dipoles 
with uniform and arbitrary magnetization direction to estimate the three components of the 
magnetic induction field produced by an arbitrary set of sources via linear inversion
of single component data. 
Results with synthetic data produced by numerical simulations and real data obtained 
on a rock sample from the Vredefort impact crater, South Africa, show the utility of the method
in the processing of magnetic microscopy data and identification of regions with largest 
concentrations of magnetic minerals.
In the second theoretical development presented in this thesis, I show that the magnetic 
moment intensity distribution on a planar and continuous layer of dipoles is all positive 
if the uniform magnetization direction of the layer is equal to that of the true sources.
Using this positivity property, I present an iterative method for estimating the uniform 
magnetization direction of a set of 3D sources by inverting total-field anomaly data.
At each iteration, the method solves a linear inverse problem to estimate a positive magnetic 
moment intensity distribution and a non-linear inverse problem to estimate the magnetization 
direction on a planar layer of dipoles. At the end of the iterative process, the uniform 
magnetization direction of the equivalent sources approximates that of the true sources.
Tests with data produced by models simulating different geological scenarios show that the method 
can be a powerful tool for estimating the uniform magnetization direction of a set of geological 
sources. Applications to airborne data over the Montes Claros de Goiás complex, located in the 
Goiás Alkaline Province, central region of Brazil, suggest that those intrusions have a strong 
remanent magnetization, in agreement with a previous independent study in the same area.

\end{foreignabstract} % resumo em inglês - deve ser utilizado por todos os tipos de documento
  \tableofcontents   % sumário - deve ser utilizado por todos os tipos de documento
  \listoffigures     % lista de figuras
  %\listoftables      % lista de tabelas

  % os dois comandos estão com problema e deverão ser
  % corrigidos futuramente
  %\printlosymbols
  %\printloabbreviations

  \mainmatter
  
  % Primeira parte do Doutorado
\part{Generalização do vínculo de positividade em camadas equivalentes magnéticas}

%\begin{foreignabstract}

In this thesis, I present two theoretical results and the applications of the 
equivalent layer for processing and interpreting magnetic data.
In the first one, I show that there is a unique planar and continuous layer of dipoles,
with a given uniform magnetization direction, that is able to reproduce, simultaneously,
the three components of the magnetic induction field produced by an arbitrary set of 
sources. This property holds true regardless of whether the magnetization direction of 
the layer is equal to the that of the sources or not.
From this theoretical result, I show that it is possible to use a planar layer of dipoles 
with uniform and arbitrary magnetization direction to estimate the three components of the 
magnetic induction field produced by an arbitrary set of sources via linear inversion
of single component data. 
Results with synthetic data produced by numerical simulations and real data obtained 
on a rock sample from the Vredefort impact crater, South Africa, show the utility of the method
in the processing of magnetic microscopy data and identification of regions with largest 
concentrations of magnetic minerals.
In the second theoretical development presented in this thesis, I show that the magnetic 
moment intensity distribution on a planar and continuous layer of dipoles is all positive 
if the uniform magnetization direction of the layer is equal to that of the true sources.
Using this positivity property, I present an iterative method for estimating the uniform 
magnetization direction of a set of 3D sources by inverting total-field anomaly data.
At each iteration, the method solves a linear inverse problem to estimate a positive magnetic 
moment intensity distribution and a non-linear inverse problem to estimate the magnetization 
direction on a planar layer of dipoles. At the end of the iterative process, the uniform 
magnetization direction of the equivalent sources approximates that of the true sources.
Tests with data produced by models simulating different geological scenarios show that the method 
can be a powerful tool for estimating the uniform magnetization direction of a set of geological 
sources. Applications to airborne data over the Montes Claros de Goiás complex, located in the 
Goiás Alkaline Province, central region of Brazil, suggest that those intrusions have a strong 
remanent magnetization, in agreement with a previous independent study in the same area.

\end{foreignabstract}   
%\include{eqlayer_I/introduction}
%\chapter{Metodologia}
\label{chap:metodologia}

\section{Estimativa das componentes e da amplitude do campo magnético}
\label{sec:componentes_campo}

Na seção \ref{sec:Gauss-Ampere}, mostrei que, se uma camada equivalente plana 
com direção de magnetização uniforme e arbitrária 
reproduz uma determinada componente $B_{\alpha}(x, y, z)$ (equação \ref{eq:B-alpha-true-generic}),
$\alpha = x, y, z$, do campo de indução magnética $\mathbf{B}(x, y, z)$ (equação \ref{eq:B-true-generic}) 
produzido por um conjunto de fontes magnéticas arbitrárias, esta camada deve, 
obrigatoriamente, reproduzir as demais componentes do campo $\mathbf{B}(x, y, z)$.
Nesta seção, apresento um método para estimar as componentes e a amplitude do campo de indução 
magnética de um conjunto de fontes magnéticas a partir da inversão de dados de uma determinada 
componente.

\subsection{Parametrização e problema direto}
\label{subsec:balpha_prob_dir}

Considere uma camada equivalente plana localizada em $z = z_{c}$, que possui uma distribuição 
contínua de intensidades de momento magnético $p(x'',y'',z_{c})$ e uma direção de magnetização 
arbitrária $\hat{\mathbf{u}}(I, D)$ (equação \ref{eq:B-alpha-layer}). 
Em situações práticas, não é possível determinar esta distribuição contínua de intensidade de momentos 
sobre a camada equivalente. 
Por esta razão, a camada é aproximada por um conjunto discreto de $M$ dipolos (fontes equivalentes) 
localizados no plano $z = z_{c}$ (Figura \ref{fig:eqlayer_balpha_sketch}).
A componente $\alpha$, $\alpha = x, y, z$, do campo de indução magnética produzida por esta camada 
discreta (componente $\alpha$ predita) no ponto $(x_{i},y_{i},z_{i})$, $i=1,\dots,N$, é dada por 
\begin{equation}
B^{\alpha}_{i} (\mathbf{p})  = \mathbf{g}^{\alpha}_{i}(\mathbf{q})^{\top} \, \mathbf{p},
\label{eq:balpha-pred-i}
\end{equation}
em que $\mathbf{q}$ é um vetor $2 \times 1$ (vetor de direção de magnetização) definido 
em termos da inclinação e declinação ($I$ e $D$) da magnetização total na camada equivalente
\begin{equation}
\mathbf{q} = \begin{bmatrix}
I \\ D 
\end{bmatrix} \: ,
\label{eq:q-vector}
\end{equation}
$\mathbf{p}$ é um vetor $M \times 1$ (vetor de momentos magnéticos) cujo $j$-ésimo elemento, $j=1,\dots,M$, 
é a intensidade do momento magnético $p_{j}$ (em $A \, m^{2}$) do $j$-ésimo dipolo e 
$\mathbf{g}^{\alpha}_{i}(\mathbf{q})$ é outro vetor $M \times 1$ cujo $j$-ésimo elemento é definido pela 
função harmônica 
\begin{equation}
g_{ij}^{\alpha}(\mathbf{q})  = \gamma_{m} \, {\mathbf{M}^{\alpha}_{ij}}^{\top} \, \hat{\mathbf{u}}(\mathbf{q}) \: .
\label{eq:g_ij-alpha}
\end{equation}
Nesta equação, $\hat{\mathbf{u}}(\mathbf{q}) \equiv \hat{\mathbf{u}}(I, D)$ é um vetor unitário
definido pela equação \ref{eq:u-hat} em função da inclinação e declinação 
da magnetização na camada equivalente ($I$ e $D$) e $\mathbf{M}^{\alpha}_{ij}$ é um vetor $3 \times 1$ 
dado por 
\begin{equation}
\mathbf{M}^{\alpha}_{ij} = \begin{bmatrix}
\partial_{\alpha x} \frac{1}{r''} \\
\partial_{\alpha y} \frac{1}{r''} \\
\partial_{\alpha z} \frac{1}{r''}
\end{bmatrix} \quad ,
\label{eq:Mij-matrix-alpha}
\end{equation}
em que $\partial_{\alpha\beta} \frac{1}{r''} \equiv \frac{\partial^{2}}{\partial \alpha \partial \beta} \frac{1}{r''}$ 
representa a segunda derivada da função $\frac{1}{r''}$ (equação \ref{eq:inv-r''}) em relação a $\alpha$ e $\beta$, 
$\alpha = x, y, z$, $\beta = x, y, z$, avaliada nas coordenadas $(x, y, z) = (x_{i}, y_{i}, z_{i})$ do $i$-ésimo dado  
observado e $(x'', y'', z_{c}) = (x_{j}, y_{j}, z_{c})$ da $j$-ésima fonte equivalente.
A componente $\alpha$ predita $B^{\alpha}_{i} (\mathbf{p})$ (equação \ref{eq:balpha-pred-i}) é 
obtida a partir da discretização da integral que define a componente $\tilde{B}_{\alpha}(x, y, z)$
(equação \ref{eq:B-alpha-layer}) produzida pela camada equivalente contínua.
Note que $B^{\alpha}_{i}(\mathbf{p})$ possui uma relação linear com o vetor de momentos 
magnéticos $\mathbf{p}$.

%% Figura
\begin{figure}[H]
	\centering
	\includegraphics[width=.7\textwidth]{Fig/mag_vec/eqlayer_figure_balpha.png}
	\caption{Representação esquemática da camada equivalente para a componente $\alpha$ do campo de 
	indução magnética. A camada é posicionada sobre o plano horizontal $z = z_{c}$. 
	$B^{\alpha}_{i}(\mathbf{p})$ é a componente $\alpha$ predita (equação \ref{eq:balpha-pred-i}) no 
	ponto $(x_{i},y_{i},z_{i})$ pelo conjunto de $M$ fontes equivalentes (pontos pretos). 
	Cada fonte é localizada em um ponto  $(x_{j},y_{j},z_{c})$, 
	$j = 1,\hdots, M$, e é representada por um dipolo de volume unitário $\upsilon_{j}$ 
	com direção de magnetização $\hat{\mathbf{u}}(\mathbf{q})$ e momento magnético $p_{j}$.}
	\label{fig:eqlayer_balpha_sketch}
\end{figure}

\subsection{Problema inverso}
\label{subsec:balpha_prob_inv}

Seja $\mathbf{B}_{z}^{o}$ o vetor de dados observados cujo $i$-ésimo elemento $B_{zi}^{o}$ é a componente vertical do campo magnético produzida por uma fonte magnética no ponto $(x_{i},y_{i},z_{i})$, $i = 1, \dots, N$. Similarmente, seja $\mathbf{B}_{z}^{p} (\mathbf{s})$ o vetor de dados preditos cujo $i$-ésimo elemento $B_{zi}^{p}(\mathbf{s})$ (equação \ref{eq:pred_data_ith-z}) é a componente vertical do campo magnético produzida por uma camada equivalente discreta no mesmo ponto $(x_{i},y_{i},z_{i})$. Com o objetivo de minimizar a diferença entre $\mathbf{B}_{z}^{o}$ e $\mathbf{B}_{z}^{p} (\mathbf{s})$, temos que resolver a equação:

\begin{equation}
\Psi(\mathbf{s}) =\lVert \mathbf{B}_{z}^{o} - \mathbf{B}_{z}^{p} (\mathbf{s}) 
	\rVert_{2}^{2} + \, \mu  \parallel \mathbf{p} \parallel_{2}^{2} \: , \\
\label{eq:goal_function_vec}
\end{equation}
em que o primeiro e o segundo termo da equação \ref{eq:goal_function_vec} são a função de ajuste e a função regularizadora de Tikhonov de ordem zero, $\mu$ é o parâmetro de regularização e $\| \cdot \|_{2}^{2}$ representa o quadrado da norma Euclidiana. 

Assumimos neste caso que a camada equivalente depende somente do vetor de momentos magnéticos $\mathbf{p}$ e, portanto, devemos impor uma direção de magnetização $\mathbf{q}$ arbitrária sobre ela. Com isso, o sistema linear que iremos resolver é dado por:

\begin{equation}
\left[ \mathbf{G}_{z}^{\top} \mathbf{G}_{z} + \mu \mathbf{I} \right] \bar{\mathbf{p}} = \mathbf{G}_{z}^{\top} \mathbf{B}_{z}^{o} \: ,
\label{eq:linear_sys_p_z}
\end{equation}
em que $\mathbf{G}_{z}$ é uma matriz de dimensão $N \times M$ cujo $ij$-ésimo elemento é dado pela função harmônica $g_{ij}^{z}(\mathbf{q})$ (equação \ref{eq:g_ij-z}) avaliada na direção de magnetização fixa $\mathbf{q}$ e $\mathbf{I}$ é uma matriz identidade de dimensão $M \times M$. A equação \ref{eq:linear_sys_p_z} é denominada como estimador de mínimos quadrados \citep{aster2005}. Após estimarmos uma distribuição de momentos magnéticos $\bar{\mathbf{p}}$ relativa a uma direção de magnetização arbitrária $\mathbf{q}$, calculamos as outras duas componentes do campo magnético aplicando a relação dada por:

\begin{equation}
\mathbf{B}_{x}^{p}  = \mathbf{G}_{x} \bar{\mathbf{p}}
\label{eq:pred_vec_x}
\end{equation}
e

\begin{equation}
\mathbf{B}_{y}^{p}  = \mathbf{G}_{y} \bar{\mathbf{p}}
\label{eq:pred_vec_y}
\end{equation}
em que $\mathbf{B}_{x}^{p}$ e $\mathbf{B}_{y}^{p}$ são, respectivamente, os vetores de dados preditos com dimensão $N \times 1$ das componentes $x$ e $y$ do campo de indução magnética. As matrizes $\mathbf{G}_{x}$ e $\mathbf{G}_{y}$ possuem dimensão $N \times M $ cujo os elementos são dados por: 

\begin{equation}
g_{ij}^{x}(\mathbf{q})  = \gamma_m \, \mathbf{M}_{ij}^{x^\top} \, \hat{\mathbf{m}}(\mathbf{q}) \: 
\label{eq:g_ij-x}
\end{equation}
e 
\begin{equation}
g_{ij}^{y}(\mathbf{q})  = \gamma_m \, \mathbf{M}_{ij}^{y^\top} \, \hat{\mathbf{m}}(\mathbf{q}) \: ,
\label{eq:g_ij-y}
\end{equation}
em que 

\begin{equation}
\mathbf{M}_{ij}^{x^\top} = \begin{bmatrix}
\partial_{xx} \frac{1}{r} & 
\partial_{xy} \frac{1}{r} &
\partial_{xz} \frac{1}{r}
\end{bmatrix}^\top \quad 
\label{eq:Mij-matrix-x}
\end{equation}
e 

\begin{equation}
\mathbf{M}_{ij}^{y^\top} = \begin{bmatrix}
\partial_{xy} \frac{1}{r} & 
\partial_{yy} \frac{1}{r} &
\partial_{yz} \frac{1}{r}
\end{bmatrix}^\top \quad .
\label{eq:Mij-matrix-y}
\end{equation}
As derivadas $\partial_{\alpha\beta} \frac{1}{r} \equiv \frac{\partial^{2}}{\partial \alpha \partial \beta} \frac{1}{r}$, representam as derivadas segundas com respeito a $\alpha = x, y$ e $\beta = x, y, z$, do inverso da distância $\frac{1}{r}$ (equação \ref{eq:inverse-distance}) entre as coordenadas de observação $(x, y, z) = (x_{i}, y_{i}, z_{i})$ e as coordenadas das fontes equivalentes $(x'', y'', z_{c}) = (x_{j}, y_{j}, z_{c})$. Além disso, calculamos a amplitude do campo magnético aplicando a relação

\begin{equation}
\mathbf{B}_a = \sqrt{ \mathbf{B}_{x}^{p^2} + \mathbf{B}_{y}^{p^2} + \mathbf{B}_{z}^{p^2}}   
\label{eq:amplitude_field}
\end{equation}
em que $\mathbf{B}_{x}^{p}$, $\mathbf{B}_{y}^{p}$ e $\mathbf{B}_{z}^{p}$ são as componentes do campo magnético e $\mathbf{B}_a$ é a vetor amplitude do campo magnético.

\section{Estimativa da direção de magnetização}
\label{sec:mag_dir_est}

Na seção \ref{sec:distribuicao-positiva}, mostrei que, para uma camada equivalente plana 
reproduzir a anomalia de campo total $\Delta T(x, y, z)$ (equação \ref{eq:Delta-T-true-mag-uniform}) 
produzida por um conjunto de fontes magnéticas com direção de magnetização uniforme, a sua distribuição 
de intensidades de momento magnético $p(x'', y'', z_{c})$ 
deve ser toda positiva (equação \ref{eq:positivity_prop}). Nesta seção, apresento um método 
iterativo que usa este vínculo de positividade dos momentos magnéticos para estimar a direção de 
magnetização uniforme das fontes magnéticas a partir da inversão de dados de anomalia de campo total.

\subsection{Parametrização e problema direto}
\label{subsec:mag_dir_prob_dir}

Considere uma camada equivalente plana localizada em $z = z_{c}$, que possui uma distribuição 
contínua de intensidades de momento magnético $p(x'',y'',z_{c})$ (equação \ref{eq:B-alpha-layer}). 
Em situações práticas, não é possível determinar esta distribuição contínua sobre a camada equivalente. 
Por esta razão, a camada é aproximada por um conjunto discreto de $M$ dipolos (fontes equivalentes) 
localizados no plano $z = z_{c}$ (Figura \ref{fig:eqlayer_tfa_sketch}). 
A anomalia de campo total produzida por esta camada 
discreta (anomalia de campo total predita) no ponto $(x_{i},y_{i},z_{i})$, $i=1,\dots,N$, é dada por 
\begin{equation}
\Delta T_{i}(\mathbf{s}) = \mathbf{g}_{i}(\mathbf{q})^{\top} \mathbf{p},
\label{eq:tfa-pred-i}
\end{equation}
em que $\mathbf{s}$ é um vetor $(M + 2) \times 1$ particionado dado por 
\begin{equation}
      \mathbf{s} = \begin{bmatrix}
		\mathbf{p} \\
		\mathbf{q}
	\end{bmatrix} \: ,
	\label{eq:s-vector}
\end{equation}
$\mathbf{q}$ é o vetor de direção de magnetização (equação \ref{eq:q-vector}), 
$\mathbf{p}$ é um vetor $M \times 1$ (vetor de momentos magnéticos) cujo $j$-ésimo elemento, $j=1,\dots,M$, 
é a intensidade do momento magnético $p_{j}$ (em $A \, m^{2}$) do $j$-ésimo dipolo e 
$\mathbf{g}_{i} (\mathbf{q})$ é outro vetor $M \times 1$ cujo $j$-ésimo elemento é definido pela 
função harmônica 
\begin{equation}
g_{ij} (\mathbf{q})  = \gamma_{m} \hat{\mathbf{u}}_{0}^T \, 
\mathbf{M}_{ij} \, \hat{\mathbf{u}}(\mathbf{q}) \: .
\label{eq:g_ij}
\end{equation}
Nesta equação, $\hat{\mathbf{u}}_{0} \equiv \hat{\mathbf{u}}(I_{0}, D_{0})$ e 
$\hat{\mathbf{u}}(\mathbf{q}) \equiv \hat{\mathbf{u}}(I, D)$ são vetores unitários
definidos pela equação \ref{eq:u-hat} em função da inclinação e declinação do 
campo principal ($I_{0}$ e $D_{0}$) e da camada equivalente ($I$ e $D$),
respectivamente, e $\mathbf{M}_{ij}$ é uma matriz $3 \times 3$ dada por 
\begin{equation}
\mathbf{M}_{ij} = \begin{bmatrix}
\partial_{xx} \frac{1}{r''} & 
\partial_{xy} \frac{1}{r''} &
\partial_{xz} \frac{1}{r''} \\
\partial_{xy} \frac{1}{r''} & 
\partial_{yy} \frac{1}{r''} &
\partial_{yz} \frac{1}{r''} \\
\partial_{xz} \frac{1}{r''} & 
\partial_{yz} \frac{1}{r''} &
\partial_{zz} \frac{1}{r''}
\end{bmatrix} \quad ,
\label{eq:Mij-matrix}
\end{equation}
em que $\partial_{\alpha\beta} \frac{1}{r''} \equiv \frac{\partial^{2}}{\partial \alpha \partial \beta} \frac{1}{r''}$ 
representa a segunda derivada da função $\frac{1}{r''}$ (equação \ref{eq:inv-r''}) em relação a $\alpha$ e $\beta$, 
$\alpha = x, y, z$ e $\beta = x, y, z$, avaliada nas coordenadas $(x, y, z) = (x_{i}, y_{i}, z_{i})$ do $i$-ésimo dado  
observado e $(x'', y'', z_{c}) = (x_{j}, y_{j}, z_{c})$ da $j$-ésima fonte equivalente.
A anomalia de campo total predita $\Delta T_{i}(\mathbf{s})$ (equação \ref{eq:tfa-pred-i}) é 
obtida substituindo-se, na equação \ref{eq:Delta-T-layer}, a 
discretização da integral que define a componente $\tilde{B}_{\alpha}(x, y, z)$
(equação \ref{eq:B-alpha-layer}) produzida pela camada equivalente contínua.
As equações $\ref{eq:tfa-pred-i}$-$\ref{eq:Mij-matrix}$ mostram que a anomalia de campo total predita 
$\Delta T_{i}(\mathbf{s})$ possui uma relação linear com o vetor de momentos magnéticos $\mathbf{p}$ e uma 
relação não linear com o vetor de direção de magnetização $\mathbf{q}$ (equação \ref{eq:q-vector}).

%% Figura 

\begin{figure}[H]
	\centering
	\includegraphics[width=.7\textwidth]{Fig/eqlayer/eqlayer_figure_tfa.png}
	\caption{Representação esquemática da camada equivalente para a anomalia de campo total. A camada é posicionada 
	sobre o plano horizontal a uma profundidade $z=z_{c}$. $\Delta T_{i}(\mathbf{s})$ é a anomalia de campo total predita 
	(equação \ref{eq:tfa-pred-i}) no ponto 
	$(x_{i},y_{i},z_{i})$ produzida pelo conjunto de $M$ fontes equivalentes (pontos pretos). Cada fonte é localizada 
	no ponto  $(x_{j},y_{j},z_{c})$, $j = 1, \dots, M$, e são representadas por um dipolo de volume unitário 
	$\upsilon_{j}$ com direção de magnetização $\hat{\mathbf{u}}(\mathbf{q})$ e momento magnético $p_{j}$. 
	$\hat{\mathbf{u}}_{0}$ é um vetor unitário na direção do campo principal.}
	\label{fig:eqlayer_tfa_sketch}
\end{figure}

\subsection{Problema inverso}
\label{subsec:mag_dir_prob_inv}

%%%%% Defining the objective function
Seja $\mathbf{\Delta T}^{o}$ o vetor de dados observados cujo $i$-ésimo elemento $\Delta T_{i}^{o}$ é a anomalia de campo total produzida por uma fonte magnética no ponto $(x_{i},y_{i},z_{i})$, $i = 1, \dots, N$. Similarmente, seja $\mathbf{\Delta T} (\mathbf{s})$ o vetor de dados preditos cujo $i$-ésimo elemento $\Delta T_{i}(\mathbf{s})$ (equação \ref{eq:tfa_pred_i}) é a anomalia de campo total produzida por uma camada equivalente discreta no mesmo ponto $(x_{i},y_{i},z_{i})$. Com o objetivo de estimarmos um vetor de parâmetros $\mathbf{s}$ (equação \ref{eq:parameter-vector}) que minimiza a diferença entre $\mathbf{\Delta T}^{o}$ e $\mathbf{\Delta T}(\mathbf{s})$, temos que resolver o problema inverso:

\begin{subequations}
	\begin{align}
	& \text{minimizar}
	& &\Psi(\mathbf{s}) =\lVert \mathbf{\Delta T}^{o} - \mathbf{\Delta T} (\mathbf{s}) 
	\rVert_{2}^{2} + \, \mu f_0 \parallel \mathbf{p} \parallel_{2}^{2} \: , \\
	& \text{sujeito a}
	& & \mathbf{p} \geqslant \mathbf{0} \: .
	\end{align}
	\label{eq:positivity_goal_function}
\end{subequations}
O primeiro e o segundo termo da equação \ref{eq:positivity_goal_function}a são, respectivamente, a função de ajuste e a função 
regularizadora de Tikhonov de ordem zero, $\mu$ é o parâmetro de regularização, $\| \cdot \|_{2}^{2}$ representa o quadrado da 
norma Euclidiana e $f_{0}$ é um fator de normalização. Na inequação \ref{eq:positivity_goal_function}b, $\mathbf{0}$ é um vetor 
$M \times 1$ com todos os elementos iguais a zero no qual o sinal da inequação é aplicado elemento a elemento. 
Este vínculo de positividade sobre o vetor de momentos magnéticos $\mathbf{p}$ é incorporado utilizando o chamado 
\textit{estimador de mínimos quadrados não negativo} ou somente NNLS (do inglês \textit{Nonnegative least squares}) proposto 
por \cite{lawson_hanson_1974}.

Para resolver este problema inverso, temos que considerar primeiramente uma expansão até segunda ordem da função objetivo 
(equação \ref{eq:positivity_goal_function}a) em torno de $\mathbf{s} = \mathbf{s}^{k}$ (equação \ref{eq:parameter-vector}):

\begin{equation}
\Psi(\mathbf{s}^{k} + \mathbf{\Delta s}^{k}) \approx \Psi(\mathbf{s}^{k}) + 
{\mathbf{J}^{k}}^{\top} \mathbf{\Delta s}^{k} + 
\frac{1}{2} {\mathbf{\Delta s}^{k}}^{\top} \mathbf{H}^{k} \mathbf{\Delta s}^{k}  \: ,
\label{eq:sec_ord_goal}
\end{equation}
em que $\mathbf{\Delta s}^{k}$ é uma perturbação no vetor de parâmetros e os termos $\mathbf{J}^{k}$ e $\mathbf{H}^{k}$ são, respectivamente, o vetor gradiente e a matriz Hessiana avaliadas em $\mathbf{s}^{k}$. Então, estimamos o vetor de perturbação $\bar{\mathbf{\Delta s}}^k$ que minimiza a função expandida (equação \ref{eq:sec_ord_goal}) tomando o seu gradiente e igualando o resultado ao vetor nulo. Este procedimento nos leva ao sistema linear 

\begin{equation}
\mathbf{H}^{k} \bar{\mathbf{\Delta s}}^{k} = - \mathbf{J}^{k} \: ,
\label{eq:linear_sys_GN}
\end{equation}
que representa o $k$-ésimo passo do método de Gauss-Newton \citep{aster2005} para a minimização da função objetivo (equação \ref{eq:positivity_goal_function}a). Reescrevemos este sistema linear desprezando as derivadas cruzadas na matriz Hessiana como: 

\begin{equation}
\left[
\begin{array}{c|c}
\mathbf{H}_{pp}^{k} & \mathbf{0} \\
\hline
\mathbf{0}^{\top} & \mathbf{H}_{qq}^{k}
\end{array}
\right] \left[ \begin{array}{c}
\bar{\mathbf{\Delta p}}^{k} \\ 
\bar{\mathbf{\Delta q}}^{k} 
\end{array} \right] \approx -\left[ \begin{array}{c}
\mathbf{J}_{p}^{k} \\ 
\mathbf{J}_{q}^{k} 
\end{array} \right] ,
\label{eq:linear_sys_GN_block}
\end{equation}
em que $\mathbf{0}$ é uma matriz $M \times 2$ que contém todos os elementos iguais a zero, $\bar{\mathbf{\Delta p}}^{k} = \bar{\mathbf{p}}^{k+1} - \bar{\mathbf{p}}^{k}$ é a correção no vetor de momentos magnéticos $\mathbf{p}$, $\bar{\mathbf{\Delta q}}^{k} = \bar{\mathbf{q}}^{k+1} - \bar{\mathbf{q}}^{k}$ é a correção no vetor direção de magnetização e os termos $\mathbf{J}_{\alpha}^{k}$ e $\mathbf{H}_{\alpha \alpha}^{k}$, $\alpha = p,q$, são os vetores gradientes e as matrizes Hessianas calculadas com respeito aos elementos dos $\mathbf{p}$ e $\mathbf{q}$, respectivamente. 

O vetor gradiente $\mathbf{J}_{p}^{k}$ e a matriz Hessiana $\mathbf{H}_{pp}^{k}$ (equação \ref{eq:linear_sys_GN_block}) relativas ao vetor de momentos magnéticos $\mathbf{p}$ (equação \ref{eq:parameter-vector}) são, respectivamente, 

\begin{equation}
\mathbf{J}_{p}^{k} = -2 {\mathbf{G}_{p}^{k}}^{\top} 
\left[ \mathbf{\Delta T}^{o} - \mathbf{\Delta T} (\bar{\mathbf{s}}^{k}) \right] + 
2\mu f_{0}^{k} \bar{\mathbf{p}}^{k} 
\label{eq:grad_p}
\end{equation}
e 

\begin{equation}
\mathbf{H}_{pp}^{k} = 2 {\mathbf{G}_{p}^{k}}^{\top} \mathbf{G}_{p}^{k} + 
2 \mu f_{0}^{k} \mathbf{I} \: ,
\label{eq:hess_p}
\end{equation}
em que $\mathbf{G}_p^{k}$ é uma matriz de dimensão $N \times M$ cujo $ij$-ésimo elemento é dado pela função harmônica $g_{ij}(\bar{\mathbf{q}}^{k})$ (equação \ref{eq:g_ij}) avaliada na direção de magnetização $\bar{\mathbf{q}}^{k}$, $\mathbf{I}$ é uma matriz identidade de dimensão $M \times M$ e $f_{0}^{k}$ é um fator de normalização igual a 

\begin{equation}
f_{0}^{k} = \dfrac{trace \left({\mathbf{G}_{p}^{k}}^{\top} \mathbf{G}_{p}^{k} \right)}{M} \, .
\label{eq:norm_factor}
\end{equation}

O vetor gradiente $\mathbf{J}_{q}^{k}$ e a matriz Hessiana $\mathbf{H}_{qq}^{k}$ (equação \ref{eq:linear_sys_GN_block}) relativas a direção de magnetização $\mathbf{q}$ (equação \ref{eq:q_vetor}) são, respectivamente, 

\begin{equation}
\mathbf{J}_{q}^{k} = -2 {\mathbf{G}_{q}^{k}}^{\top} 
\left[ \mathbf{\Delta T}^{o} - \mathbf{\Delta T} (\bar{\mathbf{s}}^{k}) \right]
\label{eq:grad_q}
\end{equation}
e

\begin{equation}
\mathbf{H}_{qq}^{k} \approx 2 {\mathbf{G}_{q}^{k}}{^\top} \mathbf{G}_{q}^{k} \: ,
\label{eq:hess_q}
\end{equation}
em que $\mathbf{G}_{q}^{k}$ é uma matriz $N \times 2$ dada por 

\begin{equation}
\mathbf{G}_{q}^{k} = \begin{bmatrix}
\partial_{I} \mathbf{g}_{1}(\bar{\mathbf{q}}^{k})^{\top} \bar{\mathbf{p}}^{k} & 
\partial_{D} \mathbf{g}_{1}(\bar{\mathbf{q}}^{k})^{\top} \bar{\mathbf{p}}^{k} \\
\vdots & \vdots  \\
\partial_{I} \mathbf{g}_{N}(\bar{\mathbf{q}}^{k})^{\top} \bar{\mathbf{p}}^{k} & 
\partial_{D} \mathbf{g}_{N}(\bar{\mathbf{q}}^{k})^{\top} \bar{\mathbf{p}}^{k} 
\end{bmatrix} \: ,
\label{eq:Gq}
\end{equation}
em que $\partial_{\alpha} \mathbf{g}_{i}(\bar{\mathbf{q}}^{k}) \equiv \frac{\partial \mathbf{g}_{i}(\bar{\mathbf{q}}^{k})}{\partial \alpha}$, $\alpha= I, D$, representa a primeira derivada do vetor $\mathbf{g}_{i}(\bar{\mathbf{q}}^{k})$ (equação \ref{eq:tfa_pred_i}) com respeito a inclinação $I$ e a declinação $D$ da magnetização total das fontes.

\subsection{Processo iterativo para a estimativa da direção de magnetização}

A iteração $k=0$ do nosso algoritmo começa com uma aproximação inicial $\bar{\mathbf{q}}^{k} = \bar{\mathbf{q}}^{0}$ para o vetor direção de magnetização $\mathbf{q}$ (equação \ref{eq:q_vetor}). Utilizando esta aproximação inicial $\bar{\mathbf{q}}^{k}$, a parte superior da equação \ref{eq:linear_sys_GN_block} nos leva ao seguinte sistema linear para o vetor de momentos magnéticos:

\begin{equation}
\left[ {\mathbf{G}_{p}^{k}}^{\top} \mathbf{G}_{p}^{k} + 
\mu f_{0}^{k} \mathbf{I} \right] \bar{\mathbf{p}}^{k} = {\mathbf{G}_{p}^{k}}^{\top} \mathbf{\Delta T}^{o} \: .
\label{eq:linear_sys_p}
\end{equation}
Para impor o vínculo de positividade (equação \ref{eq:positivity_goal_function}b) sobre a distribuição de momentos magnéticos, este sistema linear é resolvido usando o método de NNLS \citep{lawson_hanson_1974, silvadias_etal_2010}. Esta distribuição de momentos magnéticos é então usada para estimar uma correção $\bar{\mathbf{\Delta q}}^{k}$ no vetor direção de magnetização resolvendo um sistema não linear utilizando o método de Levenberg-Marquardt \citep{aster2005}:

\begin{equation}
\left[ {\mathbf{G}_{q}^{k}}^{\top} \mathbf{G}_{q}^{k} + \lambda \, \mathbf{I} \right] 
\bar{\mathbf{\Delta q}}^{k} = {\mathbf{G}_{q}^{k}}^{\top} 
\left[ \mathbf{\Delta T}^{o} - \mathbf{\Delta T} (\mathbf{s}^{k}) \right] \: ,
\label{eq:linear_sys_q}
\end{equation}
em que $\lambda$ é o parâmetro de Marquardt e $\mathbf{I}$ é uma matriz identidade. Após estimarmos a correção $\bar{\mathbf{\Delta q}}^{k}$ na $k$-ésima iteração, atualizamos a direção de magnetização aplicando a correção a seguir:

\begin{equation}
\bar{\mathbf{q}}^{k+1} = \bar{\mathbf{q}}^{k} + \bar{\mathbf{\Delta q}}^{k} \: ,
\label{eq:q_next}
\end{equation}
e utilizando esta nova direção para estimar uma nova distribuição de momentos magnéticos com a equação \ref{eq:linear_sys_p} e assim sucessivamente. O processo iterativo é interrompido quando a função objetivo (equação \ref{eq:positivity_goal_function}a) é invariante ao longo de sucessivas iterações. Mostramos também que este método falha em situações nas quais as fontes são magnetizadas verticalmente (Apêndice \ref{append:vertical-magnetization}).

\subsection{Limitação para o caso de fontes magnetizadas verticalmente}
\label{subsec:vertical-magnetization}

Nosso método falha quando a magnetização total das fontes possui a direção igual ou aproximadamente vertical. Neste apêndice, fornecemos a base teórica para o entendimento desta limitação. 

Considere o caso limite no qual a magnetização total das fontes é vertical e.g., $I = \pm 90^\circ$). Neste caso, a anomalia de campo total $\Delta T(x, y, z)$ (equação \ref{eq:tfanomaly}) não depende da declinação $D$, demonstrado pelo fato que: fontes magnetizadas verticalmente não possuem uma declinação definida. Consequentemente, o mínimo da função objetivo (equação \ref{eq:positivity_goal_function}a) não é bem definida no espaço dos parâmetros; isto é, ela é alongada na direção de $D$. Infelizmente, o vínculo de positividade sobre o vetor de momentos magnéticos (equação  \ref{eq:positivity_goal_function}b) não resolve esta ambiguidade com respeito a declinação $D$. 

Para melhor entender como esta ambiguidade afeta nosso método, começamos a analisar a matriz $\mathbf{G}_{q}^{k}$ de dimensão $N \times 2$ (equação \ref{eq:Gq}) necessária para estimar a correção $\bar{\mathbf{\Delta q}}^{k}$ para a direção de magnetização (equação \ref{eq:linear_sys_q}). Sua $i$-ésima linha é definida como o produto do vetor de momentos magnéticos estimado $\bar{\mathbf{p}}^{k}$ e as primeiras derivadas $\partial_{\alpha} \mathbf{g}_{i}(\bar{\mathbf{q}}^{k}) \equiv 
\frac{\partial \mathbf{g}_{i}(\bar{\mathbf{q}}^{k})}{\partial \alpha}$, $\alpha= I, D$, do vetor $\mathbf{g}_{i}(\mathbf{q})$ (equação \ref{eq:tfa_pred_i}), avaliada em $\mathbf{q} = \bar{\mathbf{q}}^{k}$, com respeito a inclinação $I$ e a declinação $D$ da magnetização total das fontes. O $j$-ésimo elemento $\partial_{\alpha} g_{ij}(\bar{\mathbf{q}}^{k}) \equiv 
\frac{\partial g_{ij}(\bar{\mathbf{q}}^{k})}{\partial \alpha}$ do vetor $\partial_{\alpha} \mathbf{g}_{i}(\bar{\mathbf{q}}^{k})$ de dimensão $M \times 1$ é definido como a derivada da função harmônica $g_{ij}(\mathbf{q})$ (equação \ref{eq:g_ij}) igual a 

\begin{equation}
\partial_{\alpha} g_{ij}(\bar{\mathbf{q}}^{k}) = 
\gamma_m  \hat{\mathbf{F}}_{0}^T \, \mathbf{M}_{ij} 
\partial_{\alpha} \hat{\mathbf{m}}(\bar{\mathbf{q}}^{k}) \: , \quad \alpha = I, D \: ,
\label{eq:D-alpha-gij}
\end{equation}
em que 

\begin{equation}
\partial_{I} \hat{\mathbf{m}}(\bar{\mathbf{q}}^{k}) = 
\begin{bmatrix}
	-\sin \bar{I}^{k} \cos \bar{D}^{k} \\
	-\sin \bar{I}^{k} \sin \bar{D}^{k} \\
	 \cos \bar{I}^{k}
\end{bmatrix}
\label{eq:D_mag_vec_inc}
\end{equation}
e 

\begin{equation}
\partial_{D} \hat{\mathbf{m}}(\bar{\mathbf{q}}^{k}) = 
\begin{bmatrix}
	-\cos \bar{I}^{k} \sin \bar{D}^{k} \\
	 \cos \bar{I}^{k} \cos \bar{D}^{k} \\
	 0
\end{bmatrix}
\label{eq:D_mag_vec_dec}
\end{equation}
são as derivadas do vetor unitário $\hat{\mathbf{m}}(\mathbf{q})$ (equação \ref{eq:mag_vet}), avaliadas na direção de magnetização $\bar{\mathbf{q}}^{k} = \left[ \bar{I}^{k} \:\: \bar{D}^{k} \right]^{\top}$, com respeito a $I$ e $D$. 
 
Note que, quando a inclinação estimada $\bar{I}^{k}$ se aproxima de $\pm 90^{\circ}$, todos os elementos que formam o vetor  $\partial_{D} \hat{\mathbf{m}}(\bar{\mathbf{q}}^{k})$ (equação \ref{eq:D_mag_vec_dec})e, consequentemente, a segunda coluna da matriz $\mathbf{G}_{q}^{k}$ (equação \ref{eq:Gq}) tendem a zero. Como consequência, o problema não-linear para estimar a direção de magnetização (equação \ref{eq:linear_sys_q}) não é sensível a mudanças na declinação $D$ e a convergência do nosso método é muito lenta devido a suavidade da função objetivo $\Psi(\mathbf{s})$ (equação \ref{eq:positivity_goal_function}a) no espaço de parâmetros. 


\section{Profundidade da camada ($\mathbf{z_{c}}$) e parâmetro de regularização ($\mathbf{\mu}$)}

O procedimento pelo qual utilizamos a camada equivalente para estimar a direção de magnetização total das fontes magnéticas e o cálculo das componentes do campo magnético requer a escolha de dois parâmetros principais. O primeiro é a profundidade da camada $z_c$ (Figuras \ref{fig:eqlayer_tfa_sketch} e \ref{fig:eqlayer_bz_sketch}) e o segundo é o parâmetro de regularização $\mu$ mostrado na equação \ref{eq:linear_sys_p}. 

O método utilizado para a escolha da profundidade da camada é baseado na abordagem clássica proposta por \cite{dampney1969}. O autor aponta que o posicionamento da camada deve satisfazer um intervalo de $2,5$ a $6,0$ vezes o espaçamento dos dados. Vale ressaltar que esta regra foi aplicada pelos autores em uma grade com dados regularmente espaçados. Contudo, a escolha para aplicar nosso método corresponde a um intervalo de $2$ a $3$ vezes o valor do maior espaçamento entre os dados. É necessário lembrar que este intervalo foi encontrado empiricamente. 

Para resolver a equação \ref{eq:linear_sys_p}, temos que escolher um valor confiável para o parâmetro de regularização. Com este propósito, usamos o método da curva-L, que serve como uma filtragem de ruídos dos dados, sem que o resultado final perca informações. O 'cotovelo' desta curva é o valor ótimo de parâmetro no qual é feito o balanço entre a função de ajuste e a função regularizadora. 
%\include{eqlayer_I/simulations}
%\include{eqlayer_I/real_data}
%\include{eqlayer_I/conclusions}
 
% Segunda parte do Doutorado
\part{Cálculo do vetor magnético a partir de dados de microscopia magnética no domínio do espaço}
%\chapter{Introdução}
\label{chap:introducao}

A maioria dos métodos magnéticos requerem o conhecimento da direção de magnetização, caso contrário produzem informações insatisfatórias sobre as fontes. Este fato tem impulsionado o desenvolvimento de diversas técnicas para estimar a direção de magnetização ao longo dos últimos 50 anos. As estratégias para estimar esta quantidade podem ser dividida em dois grupos principais. O primeiro grupo compreende aqueles métodos que presumem informações prévias acerca da geometria das fontes. O método iterativo apresentado por \cite{bhattacharyya1966} presume que as fontes magnéticas tem formato de prismas retangulares. \cite{emilia_massey_1974} aproxima um monte submarino por um conjunto de prismas justapostos com direção de magnetização uniforme e intensidade variável. \cite{parker_etal_1987} aproximam a geometria de um monte submarion utilizando uma cobertura de faces triangulares e estimam a magnetização interna próxima a uma solução uniforme. \cite{parker_etal_1987} apresentaram um método que estima a direção de magnetização total e a orientação espacial de uma fonte isolada que possui três planos ortogonais de simetria. \cite{kubota2005} aproximaram também um monte submarino por um conjunto de prismas justapostos, mas estimaram a direção de magnetização para cada um dos prismas. Finalmente, \cite{oliveirajr_etal_2015} aproximam as fontes magnéticas por corpos esféricos de centros conhecidos e estimam suas direções de magnetização. O segundo grupo é formado pelos métodos que não presumem informações sobre a geometria das fontes. \cite{fedi_etal_1994}, por exemplo, propôs um método que determina a melhor direção de magnetização dentre um conjunto de tentativas usadas para realizar sucessivas reduções ao polo no domínio de Fourier. \cite{phillips2005} utilizaram integrais de Helbig para estimar as componentes do vetor de momento magnético. \cite{tontini_pedersen_2008} extenderam o método de Phillips utilizando as mesmas integrais de Helbig para estimar a direção de magnetização e sua magnitude, e fornecendo informações sobre a posição do centro de distribuição de magnetização. \cite{lelievre_oldenburg_2009} desenvolveram um método para estimar a direção de magnetização em cenários geológicos complexos. Este método aproxima a subsuperfície por um grid de prismas justapostos e estima as componentes do vetor magnetização para cada prisma. Além destes métodos, existem aqueles que são baseados na correlação de quantidades potenciais \citep[e.g.,][]{dannemiller_li_2006,gerovska_etal_2009,liu_etal_2015,zhang_etal_2018}. 

Estimar a direção de magnetização é extremamente importante não só para interpretação, mas também para o processamento de anomalia de campo total. Uma técnica no domínio do espaço comumente utilizada para esta finalidade é a camada equivalente. Esta técnica foi introduzida na geofísica de exploração por \cite{dampney1969} e \cite{emilia_massey_1974} para o processamento de dados gravimétricos e magnéticos, respectivamente. Após estes trabalhos pioneiros, esta técnica tem sido vastamente utilizada para realizar interpolação \citep{cordell_1992, mendonca-silva_1994, barnes-lumley_2011, siqueira_etal_2017}, continuação para cima (ou para baixo) \citep{hansen-miyazaki_1984, li-oldenburg_2010}, redução ao polo \citep{silva_1986, leao-silva_1989, guspi-novara_2009, oliveirajr-etal_2013}, calcular a amplitude do campo anômalo \citep{li_li_2014} e para filtrar ruídos de dados de gradiometria \citep{martinez_li_2016}. A técnica da camada equivalente consiste em aproximar um conjunto de dados observados por dados produzidos por uma camada composta de fontes discretas (e.g., prismas, dipolos ou pontos de massa), que são comumente conhecidas como fontes equivalentes. Os dados produzidos por esta camada fictícia (a camada equivalente) são chamados de dados preditos. 

Em microscopia magnética, a técnica da camada equivalente é geralmente utilizada para a interpretação da distribuição de momentos magnéticos em uma lâmina de rocha. Note que, neste caso, a camada equivalente se assemelha a fonte verdadeira (a lâmina de rocha). \cite{weiss2007} apresentou um dos primeiros trabalhos utilizando a camada equivalente em microscopia magnética. Os autores apontaram que a distribuição de magnetização é interamente positiva se a direção de magnetização das fontes equivalentes é igual a direção utilizada para uma amostra de rocha magnetizada artificialmente. \cite{baratchart2013} mostraram matematicamente que, assumindo uma direção de magnetização uniforme, o problema inverso para estimar a distribuição de momentos magnéticos é único. \cite{lima2013} propôs um método no domínio da frequência para investigar soluções tendo a direção de magnetização uniforme iguais a das lâminas de rocha. Eles mostraram empiricamente que, neste caso, a distribuição de momentos magnéticos na camada é inteiramente positiva.  Em geofísica de exploração, a camada equivalente é predominantemente utilizada para o processamento de dados potenciais. Neste sentido, não existe relação entre a distribuição de propriedade física estimada sobre a camada equivalente e as fontes geológicas verdadeiras. Poucos autores tem direcionado o uso da camada equivalente para a interpretação de fontes geológicas. \cite{pedersen1991}, por exemplo, discutiu a relaçao entre o campo potencial e a fonte equivalente. \cite{medeiros_silva1996} e \cite{silvadias_etal_2010} estimaram um mapa de magnetização aparente sobre a camada utilizando regularizações de Tikhonov e entrópica, respectivamente. \cite{siqueira_etal_2017} estabeleceu a relação entre o excesso de massa estimado sobre a camada e o verdadeiro. \cite{li_etal_2014} provaram, utilizando uma abordagem no domínio de Fourier, a existência de uma distribuição de momentos magnéticos positiva sobre a camada e utilizaram esta propriedade para contornar o problema de instabilidade em baixas latitudes. No entanto, estes autores consideraram somente um caso particular no qual as fontes magnéticas tenham magnetização puramente induzida. 

Neste trabalho, provamos matematicamente que a distribuição de momentos magnéticos positiva sobre a camada equivalente existe mesmo que a magnetização das fontes verdadeiras seja remanente. Esta distribuição positiva de momentos magnéticos é válida para todos os casos nos quais a direção de magnetização das fontes equivalentes tem a mesma direção das fontes verdadeiras, mesmo que a magnetização destas fontes seja puramente induzida ou não. Amparado neste vínculo de positividade generalizado, apresentamos um método iterativo que usa a técnica da camada equivalente para estimar a direção de magnetização uniforme de fontes arbitrárias invertendo dados de anomalia de campo total. Nosso método não presume qualquer informação sobre a geometria das fontes. A cada iteração nosso método resolve (1) um problema linear, sujeito a um vínculo de positividade para estimar a distribuição de momentos magnéticos sobre uma camada de dipolos e (2) um problema inverso não-linear para estimar a direção de magnetização uniforme das fontes equivalentes. Testes com dados sintéticos gerados por cenários geológicos diferentes mostram que a direção de magnetização estimada converge para a direção verdadeira das fontes verdadeiras. Aplicamos também nosso método a dados de campo provenientes da província alcalina de Goiás, sobre o complexo de Montes Claros, na região central do Brasil. Nosso resultado está de acordo com os resultados obtidos por \cite{zhang_etal_2018} para a mesma área, sugerindo a presença de marcantes componentes de magnetização remanente e mostrando a boa performance do nosso método na interpretação de cenários geológicos complexos. 



%\begin{abstract}

Nesta tese, apresento dois resultados teóricos e as respectivas aplicações da 
camada equivalente no processamento e interpretação de dados magnéticos. 
No primeiro deles, mostro que há uma única camada plana e contínua de dipolos, com uma 
determinada direção de magnetização uniforme, capaz de reproduzir, simultaneamente, as 
três componentes do campo de indução magnética produzido por um conjunto arbitrário de fontes.
Esta propriedade é válida independentemente se a direção de magnetização na camada é igual a 
das fontes ou não.
A partir deste resultado teórico, mostro que é possível usar uma camada plana de dipolos com 
direção de magnetização uniforme e arbitrária para estimar as três componentes do campo de 
indução magnética produzido por um conjunto arbitrário de fontes via inversão linear 
de dados de uma única componente.
Resultados com dados sintéticos produzidos por simulações numéricas e dados reais obtidos 
sobre uma amostra de rocha proveniente da cratera de Vredefort, África do Sul,
mostram a utilidade do método no processamento de dados de microscopia magnética e na 
identificação de regiões com maior concentração de minerais magnéticos.
No segundo desenvolvimento teórico apresentado nesta tese, mostro que a distribuição de 
intensidades de momento magnético sobre uma camada plana e contínua de dipolos é toda 
positiva se a direção de magnetização uniforme na camada é igual àquela das fontes verdadeiras.
Usando esta propriedade de positividade, apresento um método iterativo para estimar a direção 
de magnetização uniforme de um conjunto de fontes 3D a partir da inversão de dados de anomalia de 
campo total. A cada iteração, o método resolve um 
problema inverso linear para estimar uma distribuição de intensidades de momento magnético positiva
e um problema inverso não-linear para estimar a direção de magnetização sobre uma camada plana de dipolos.
Ao final do processo, a direção de magnetização uniforme das fontes equivalentes se aproxima daquela 
das fontes verdadeiras.
Testes com dados produzidos por modelos que simulam diferentes cenários geológicos mostram que o 
método pode ser uma ferramenta poderosa para estimar a direção de magnetização uniforme 
de um conjunto de fontes geológicas. Aplicações a dados de aerolevantamento sobre o complexo de Montes Claros
de Goiás, localizado na província alcalina de Goiás, região central do Brasil, sugerem que estas intrusões 
possuem intensa magnetização remanente, o que está em acordo com um estudo independente conduzido 
previamente na mesma área. 

\end{abstract}   
%\chapter{Introdução}
\label{chap:introducao}

A maioria dos métodos magnéticos requerem o conhecimento da direção de magnetização, caso contrário produzem informações insatisfatórias sobre as fontes. Este fato tem impulsionado o desenvolvimento de diversas técnicas para estimar a direção de magnetização ao longo dos últimos 50 anos. As estratégias para estimar esta quantidade podem ser dividida em dois grupos principais. O primeiro grupo compreende aqueles métodos que presumem informações prévias acerca da geometria das fontes. O método iterativo apresentado por \cite{bhattacharyya1966} presume que as fontes magnéticas tem formato de prismas retangulares. \cite{emilia_massey_1974} aproxima um monte submarino por um conjunto de prismas justapostos com direção de magnetização uniforme e intensidade variável. \cite{parker_etal_1987} aproximam a geometria de um monte submarion utilizando uma cobertura de faces triangulares e estimam a magnetização interna próxima a uma solução uniforme. \cite{parker_etal_1987} apresentaram um método que estima a direção de magnetização total e a orientação espacial de uma fonte isolada que possui três planos ortogonais de simetria. \cite{kubota2005} aproximaram também um monte submarino por um conjunto de prismas justapostos, mas estimaram a direção de magnetização para cada um dos prismas. Finalmente, \cite{oliveirajr_etal_2015} aproximam as fontes magnéticas por corpos esféricos de centros conhecidos e estimam suas direções de magnetização. O segundo grupo é formado pelos métodos que não presumem informações sobre a geometria das fontes. \cite{fedi_etal_1994}, por exemplo, propôs um método que determina a melhor direção de magnetização dentre um conjunto de tentativas usadas para realizar sucessivas reduções ao polo no domínio de Fourier. \cite{phillips2005} utilizaram integrais de Helbig para estimar as componentes do vetor de momento magnético. \cite{tontini_pedersen_2008} extenderam o método de Phillips utilizando as mesmas integrais de Helbig para estimar a direção de magnetização e sua magnitude, e fornecendo informações sobre a posição do centro de distribuição de magnetização. \cite{lelievre_oldenburg_2009} desenvolveram um método para estimar a direção de magnetização em cenários geológicos complexos. Este método aproxima a subsuperfície por um grid de prismas justapostos e estima as componentes do vetor magnetização para cada prisma. Além destes métodos, existem aqueles que são baseados na correlação de quantidades potenciais \citep[e.g.,][]{dannemiller_li_2006,gerovska_etal_2009,liu_etal_2015,zhang_etal_2018}. 

Estimar a direção de magnetização é extremamente importante não só para interpretação, mas também para o processamento de anomalia de campo total. Uma técnica no domínio do espaço comumente utilizada para esta finalidade é a camada equivalente. Esta técnica foi introduzida na geofísica de exploração por \cite{dampney1969} e \cite{emilia_massey_1974} para o processamento de dados gravimétricos e magnéticos, respectivamente. Após estes trabalhos pioneiros, esta técnica tem sido vastamente utilizada para realizar interpolação \citep{cordell_1992, mendonca-silva_1994, barnes-lumley_2011, siqueira_etal_2017}, continuação para cima (ou para baixo) \citep{hansen-miyazaki_1984, li-oldenburg_2010}, redução ao polo \citep{silva_1986, leao-silva_1989, guspi-novara_2009, oliveirajr-etal_2013}, calcular a amplitude do campo anômalo \citep{li_li_2014} e para filtrar ruídos de dados de gradiometria \citep{martinez_li_2016}. A técnica da camada equivalente consiste em aproximar um conjunto de dados observados por dados produzidos por uma camada composta de fontes discretas (e.g., prismas, dipolos ou pontos de massa), que são comumente conhecidas como fontes equivalentes. Os dados produzidos por esta camada fictícia (a camada equivalente) são chamados de dados preditos. 

Em microscopia magnética, a técnica da camada equivalente é geralmente utilizada para a interpretação da distribuição de momentos magnéticos em uma lâmina de rocha. Note que, neste caso, a camada equivalente se assemelha a fonte verdadeira (a lâmina de rocha). \cite{weiss2007} apresentou um dos primeiros trabalhos utilizando a camada equivalente em microscopia magnética. Os autores apontaram que a distribuição de magnetização é interamente positiva se a direção de magnetização das fontes equivalentes é igual a direção utilizada para uma amostra de rocha magnetizada artificialmente. \cite{baratchart2013} mostraram matematicamente que, assumindo uma direção de magnetização uniforme, o problema inverso para estimar a distribuição de momentos magnéticos é único. \cite{lima2013} propôs um método no domínio da frequência para investigar soluções tendo a direção de magnetização uniforme iguais a das lâminas de rocha. Eles mostraram empiricamente que, neste caso, a distribuição de momentos magnéticos na camada é inteiramente positiva.  Em geofísica de exploração, a camada equivalente é predominantemente utilizada para o processamento de dados potenciais. Neste sentido, não existe relação entre a distribuição de propriedade física estimada sobre a camada equivalente e as fontes geológicas verdadeiras. Poucos autores tem direcionado o uso da camada equivalente para a interpretação de fontes geológicas. \cite{pedersen1991}, por exemplo, discutiu a relaçao entre o campo potencial e a fonte equivalente. \cite{medeiros_silva1996} e \cite{silvadias_etal_2010} estimaram um mapa de magnetização aparente sobre a camada utilizando regularizações de Tikhonov e entrópica, respectivamente. \cite{siqueira_etal_2017} estabeleceu a relação entre o excesso de massa estimado sobre a camada e o verdadeiro. \cite{li_etal_2014} provaram, utilizando uma abordagem no domínio de Fourier, a existência de uma distribuição de momentos magnéticos positiva sobre a camada e utilizaram esta propriedade para contornar o problema de instabilidade em baixas latitudes. No entanto, estes autores consideraram somente um caso particular no qual as fontes magnéticas tenham magnetização puramente induzida. 

Neste trabalho, provamos matematicamente que a distribuição de momentos magnéticos positiva sobre a camada equivalente existe mesmo que a magnetização das fontes verdadeiras seja remanente. Esta distribuição positiva de momentos magnéticos é válida para todos os casos nos quais a direção de magnetização das fontes equivalentes tem a mesma direção das fontes verdadeiras, mesmo que a magnetização destas fontes seja puramente induzida ou não. Amparado neste vínculo de positividade generalizado, apresentamos um método iterativo que usa a técnica da camada equivalente para estimar a direção de magnetização uniforme de fontes arbitrárias invertendo dados de anomalia de campo total. Nosso método não presume qualquer informação sobre a geometria das fontes. A cada iteração nosso método resolve (1) um problema linear, sujeito a um vínculo de positividade para estimar a distribuição de momentos magnéticos sobre uma camada de dipolos e (2) um problema inverso não-linear para estimar a direção de magnetização uniforme das fontes equivalentes. Testes com dados sintéticos gerados por cenários geológicos diferentes mostram que a direção de magnetização estimada converge para a direção verdadeira das fontes verdadeiras. Aplicamos também nosso método a dados de campo provenientes da província alcalina de Goiás, sobre o complexo de Montes Claros, na região central do Brasil. Nosso resultado está de acordo com os resultados obtidos por \cite{zhang_etal_2018} para a mesma área, sugerindo a presença de marcantes componentes de magnetização remanente e mostrando a boa performance do nosso método na interpretação de cenários geológicos complexos. 



\chapter{Metodologia}
\label{chap:metodologia}

\section{Estimativa das componentes e da amplitude do campo magnético}
\label{sec:componentes_campo}

Na seção \ref{sec:Gauss-Ampere}, mostrei que, se uma camada equivalente plana 
com direção de magnetização uniforme e arbitrária 
reproduz uma determinada componente $B_{\alpha}(x, y, z)$ (equação \ref{eq:B-alpha-true-generic}),
$\alpha = x, y, z$, do campo de indução magnética $\mathbf{B}(x, y, z)$ (equação \ref{eq:B-true-generic}) 
produzido por um conjunto de fontes magnéticas arbitrárias, esta camada deve, 
obrigatoriamente, reproduzir as demais componentes do campo $\mathbf{B}(x, y, z)$.
Nesta seção, apresento um método para estimar as componentes e a amplitude do campo de indução 
magnética de um conjunto de fontes magnéticas a partir da inversão de dados de uma determinada 
componente.

\subsection{Parametrização e problema direto}
\label{subsec:balpha_prob_dir}

Considere uma camada equivalente plana localizada em $z = z_{c}$, que possui uma distribuição 
contínua de intensidades de momento magnético $p(x'',y'',z_{c})$ e uma direção de magnetização 
arbitrária $\hat{\mathbf{u}}(I, D)$ (equação \ref{eq:B-alpha-layer}). 
Em situações práticas, não é possível determinar esta distribuição contínua de intensidade de momentos 
sobre a camada equivalente. 
Por esta razão, a camada é aproximada por um conjunto discreto de $M$ dipolos (fontes equivalentes) 
localizados no plano $z = z_{c}$ (Figura \ref{fig:eqlayer_balpha_sketch}).
A componente $\alpha$, $\alpha = x, y, z$, do campo de indução magnética produzida por esta camada 
discreta (componente $\alpha$ predita) no ponto $(x_{i},y_{i},z_{i})$, $i=1,\dots,N$, é dada por 
\begin{equation}
B^{\alpha}_{i} (\mathbf{p})  = \mathbf{g}^{\alpha}_{i}(\mathbf{q})^{\top} \, \mathbf{p},
\label{eq:balpha-pred-i}
\end{equation}
em que $\mathbf{q}$ é um vetor $2 \times 1$ (vetor de direção de magnetização) definido 
em termos da inclinação e declinação ($I$ e $D$) da magnetização total na camada equivalente
\begin{equation}
\mathbf{q} = \begin{bmatrix}
I \\ D 
\end{bmatrix} \: ,
\label{eq:q-vector}
\end{equation}
$\mathbf{p}$ é um vetor $M \times 1$ (vetor de momentos magnéticos) cujo $j$-ésimo elemento, $j=1,\dots,M$, 
é a intensidade do momento magnético $p_{j}$ (em $A \, m^{2}$) do $j$-ésimo dipolo e 
$\mathbf{g}^{\alpha}_{i}(\mathbf{q})$ é outro vetor $M \times 1$ cujo $j$-ésimo elemento é definido pela 
função harmônica 
\begin{equation}
g_{ij}^{\alpha}(\mathbf{q})  = \gamma_{m} \, {\mathbf{M}^{\alpha}_{ij}}^{\top} \, \hat{\mathbf{u}}(\mathbf{q}) \: .
\label{eq:g_ij-alpha}
\end{equation}
Nesta equação, $\hat{\mathbf{u}}(\mathbf{q}) \equiv \hat{\mathbf{u}}(I, D)$ é um vetor unitário
definido pela equação \ref{eq:u-hat} em função da inclinação e declinação 
da magnetização na camada equivalente ($I$ e $D$) e $\mathbf{M}^{\alpha}_{ij}$ é um vetor $3 \times 1$ 
dado por 
\begin{equation}
\mathbf{M}^{\alpha}_{ij} = \begin{bmatrix}
\partial_{\alpha x} \frac{1}{r''} \\
\partial_{\alpha y} \frac{1}{r''} \\
\partial_{\alpha z} \frac{1}{r''}
\end{bmatrix} \quad ,
\label{eq:Mij-matrix-alpha}
\end{equation}
em que $\partial_{\alpha\beta} \frac{1}{r''} \equiv \frac{\partial^{2}}{\partial \alpha \partial \beta} \frac{1}{r''}$ 
representa a segunda derivada da função $\frac{1}{r''}$ (equação \ref{eq:inv-r''}) em relação a $\alpha$ e $\beta$, 
$\alpha = x, y, z$, $\beta = x, y, z$, avaliada nas coordenadas $(x, y, z) = (x_{i}, y_{i}, z_{i})$ do $i$-ésimo dado  
observado e $(x'', y'', z_{c}) = (x_{j}, y_{j}, z_{c})$ da $j$-ésima fonte equivalente.
A componente $\alpha$ predita $B^{\alpha}_{i} (\mathbf{p})$ (equação \ref{eq:balpha-pred-i}) é 
obtida a partir da discretização da integral que define a componente $\tilde{B}_{\alpha}(x, y, z)$
(equação \ref{eq:B-alpha-layer}) produzida pela camada equivalente contínua.
Note que $B^{\alpha}_{i}(\mathbf{p})$ possui uma relação linear com o vetor de momentos 
magnéticos $\mathbf{p}$.

%% Figura
\begin{figure}[H]
	\centering
	\includegraphics[width=.7\textwidth]{Fig/mag_vec/eqlayer_figure_balpha.png}
	\caption{Representação esquemática da camada equivalente para a componente $\alpha$ do campo de 
	indução magnética. A camada é posicionada sobre o plano horizontal $z = z_{c}$. 
	$B^{\alpha}_{i}(\mathbf{p})$ é a componente $\alpha$ predita (equação \ref{eq:balpha-pred-i}) no 
	ponto $(x_{i},y_{i},z_{i})$ pelo conjunto de $M$ fontes equivalentes (pontos pretos). 
	Cada fonte é localizada em um ponto  $(x_{j},y_{j},z_{c})$, 
	$j = 1,\hdots, M$, e é representada por um dipolo de volume unitário $\upsilon_{j}$ 
	com direção de magnetização $\hat{\mathbf{u}}(\mathbf{q})$ e momento magnético $p_{j}$.}
	\label{fig:eqlayer_balpha_sketch}
\end{figure}

\subsection{Problema inverso}
\label{subsec:balpha_prob_inv}

Seja $\mathbf{B}_{z}^{o}$ o vetor de dados observados cujo $i$-ésimo elemento $B_{zi}^{o}$ é a componente vertical do campo magnético produzida por uma fonte magnética no ponto $(x_{i},y_{i},z_{i})$, $i = 1, \dots, N$. Similarmente, seja $\mathbf{B}_{z}^{p} (\mathbf{s})$ o vetor de dados preditos cujo $i$-ésimo elemento $B_{zi}^{p}(\mathbf{s})$ (equação \ref{eq:pred_data_ith-z}) é a componente vertical do campo magnético produzida por uma camada equivalente discreta no mesmo ponto $(x_{i},y_{i},z_{i})$. Com o objetivo de minimizar a diferença entre $\mathbf{B}_{z}^{o}$ e $\mathbf{B}_{z}^{p} (\mathbf{s})$, temos que resolver a equação:

\begin{equation}
\Psi(\mathbf{s}) =\lVert \mathbf{B}_{z}^{o} - \mathbf{B}_{z}^{p} (\mathbf{s}) 
	\rVert_{2}^{2} + \, \mu  \parallel \mathbf{p} \parallel_{2}^{2} \: , \\
\label{eq:goal_function_vec}
\end{equation}
em que o primeiro e o segundo termo da equação \ref{eq:goal_function_vec} são a função de ajuste e a função regularizadora de Tikhonov de ordem zero, $\mu$ é o parâmetro de regularização e $\| \cdot \|_{2}^{2}$ representa o quadrado da norma Euclidiana. 

Assumimos neste caso que a camada equivalente depende somente do vetor de momentos magnéticos $\mathbf{p}$ e, portanto, devemos impor uma direção de magnetização $\mathbf{q}$ arbitrária sobre ela. Com isso, o sistema linear que iremos resolver é dado por:

\begin{equation}
\left[ \mathbf{G}_{z}^{\top} \mathbf{G}_{z} + \mu \mathbf{I} \right] \bar{\mathbf{p}} = \mathbf{G}_{z}^{\top} \mathbf{B}_{z}^{o} \: ,
\label{eq:linear_sys_p_z}
\end{equation}
em que $\mathbf{G}_{z}$ é uma matriz de dimensão $N \times M$ cujo $ij$-ésimo elemento é dado pela função harmônica $g_{ij}^{z}(\mathbf{q})$ (equação \ref{eq:g_ij-z}) avaliada na direção de magnetização fixa $\mathbf{q}$ e $\mathbf{I}$ é uma matriz identidade de dimensão $M \times M$. A equação \ref{eq:linear_sys_p_z} é denominada como estimador de mínimos quadrados \citep{aster2005}. Após estimarmos uma distribuição de momentos magnéticos $\bar{\mathbf{p}}$ relativa a uma direção de magnetização arbitrária $\mathbf{q}$, calculamos as outras duas componentes do campo magnético aplicando a relação dada por:

\begin{equation}
\mathbf{B}_{x}^{p}  = \mathbf{G}_{x} \bar{\mathbf{p}}
\label{eq:pred_vec_x}
\end{equation}
e

\begin{equation}
\mathbf{B}_{y}^{p}  = \mathbf{G}_{y} \bar{\mathbf{p}}
\label{eq:pred_vec_y}
\end{equation}
em que $\mathbf{B}_{x}^{p}$ e $\mathbf{B}_{y}^{p}$ são, respectivamente, os vetores de dados preditos com dimensão $N \times 1$ das componentes $x$ e $y$ do campo de indução magnética. As matrizes $\mathbf{G}_{x}$ e $\mathbf{G}_{y}$ possuem dimensão $N \times M $ cujo os elementos são dados por: 

\begin{equation}
g_{ij}^{x}(\mathbf{q})  = \gamma_m \, \mathbf{M}_{ij}^{x^\top} \, \hat{\mathbf{m}}(\mathbf{q}) \: 
\label{eq:g_ij-x}
\end{equation}
e 
\begin{equation}
g_{ij}^{y}(\mathbf{q})  = \gamma_m \, \mathbf{M}_{ij}^{y^\top} \, \hat{\mathbf{m}}(\mathbf{q}) \: ,
\label{eq:g_ij-y}
\end{equation}
em que 

\begin{equation}
\mathbf{M}_{ij}^{x^\top} = \begin{bmatrix}
\partial_{xx} \frac{1}{r} & 
\partial_{xy} \frac{1}{r} &
\partial_{xz} \frac{1}{r}
\end{bmatrix}^\top \quad 
\label{eq:Mij-matrix-x}
\end{equation}
e 

\begin{equation}
\mathbf{M}_{ij}^{y^\top} = \begin{bmatrix}
\partial_{xy} \frac{1}{r} & 
\partial_{yy} \frac{1}{r} &
\partial_{yz} \frac{1}{r}
\end{bmatrix}^\top \quad .
\label{eq:Mij-matrix-y}
\end{equation}
As derivadas $\partial_{\alpha\beta} \frac{1}{r} \equiv \frac{\partial^{2}}{\partial \alpha \partial \beta} \frac{1}{r}$, representam as derivadas segundas com respeito a $\alpha = x, y$ e $\beta = x, y, z$, do inverso da distância $\frac{1}{r}$ (equação \ref{eq:inverse-distance}) entre as coordenadas de observação $(x, y, z) = (x_{i}, y_{i}, z_{i})$ e as coordenadas das fontes equivalentes $(x'', y'', z_{c}) = (x_{j}, y_{j}, z_{c})$. Além disso, calculamos a amplitude do campo magnético aplicando a relação

\begin{equation}
\mathbf{B}_a = \sqrt{ \mathbf{B}_{x}^{p^2} + \mathbf{B}_{y}^{p^2} + \mathbf{B}_{z}^{p^2}}   
\label{eq:amplitude_field}
\end{equation}
em que $\mathbf{B}_{x}^{p}$, $\mathbf{B}_{y}^{p}$ e $\mathbf{B}_{z}^{p}$ são as componentes do campo magnético e $\mathbf{B}_a$ é a vetor amplitude do campo magnético.

\section{Estimativa da direção de magnetização}
\label{sec:mag_dir_est}

Na seção \ref{sec:distribuicao-positiva}, mostrei que, para uma camada equivalente plana 
reproduzir a anomalia de campo total $\Delta T(x, y, z)$ (equação \ref{eq:Delta-T-true-mag-uniform}) 
produzida por um conjunto de fontes magnéticas com direção de magnetização uniforme, a sua distribuição 
de intensidades de momento magnético $p(x'', y'', z_{c})$ 
deve ser toda positiva (equação \ref{eq:positivity_prop}). Nesta seção, apresento um método 
iterativo que usa este vínculo de positividade dos momentos magnéticos para estimar a direção de 
magnetização uniforme das fontes magnéticas a partir da inversão de dados de anomalia de campo total.

\subsection{Parametrização e problema direto}
\label{subsec:mag_dir_prob_dir}

Considere uma camada equivalente plana localizada em $z = z_{c}$, que possui uma distribuição 
contínua de intensidades de momento magnético $p(x'',y'',z_{c})$ (equação \ref{eq:B-alpha-layer}). 
Em situações práticas, não é possível determinar esta distribuição contínua sobre a camada equivalente. 
Por esta razão, a camada é aproximada por um conjunto discreto de $M$ dipolos (fontes equivalentes) 
localizados no plano $z = z_{c}$ (Figura \ref{fig:eqlayer_tfa_sketch}). 
A anomalia de campo total produzida por esta camada 
discreta (anomalia de campo total predita) no ponto $(x_{i},y_{i},z_{i})$, $i=1,\dots,N$, é dada por 
\begin{equation}
\Delta T_{i}(\mathbf{s}) = \mathbf{g}_{i}(\mathbf{q})^{\top} \mathbf{p},
\label{eq:tfa-pred-i}
\end{equation}
em que $\mathbf{s}$ é um vetor $(M + 2) \times 1$ particionado dado por 
\begin{equation}
      \mathbf{s} = \begin{bmatrix}
		\mathbf{p} \\
		\mathbf{q}
	\end{bmatrix} \: ,
	\label{eq:s-vector}
\end{equation}
$\mathbf{q}$ é o vetor de direção de magnetização (equação \ref{eq:q-vector}), 
$\mathbf{p}$ é um vetor $M \times 1$ (vetor de momentos magnéticos) cujo $j$-ésimo elemento, $j=1,\dots,M$, 
é a intensidade do momento magnético $p_{j}$ (em $A \, m^{2}$) do $j$-ésimo dipolo e 
$\mathbf{g}_{i} (\mathbf{q})$ é outro vetor $M \times 1$ cujo $j$-ésimo elemento é definido pela 
função harmônica 
\begin{equation}
g_{ij} (\mathbf{q})  = \gamma_{m} \hat{\mathbf{u}}_{0}^T \, 
\mathbf{M}_{ij} \, \hat{\mathbf{u}}(\mathbf{q}) \: .
\label{eq:g_ij}
\end{equation}
Nesta equação, $\hat{\mathbf{u}}_{0} \equiv \hat{\mathbf{u}}(I_{0}, D_{0})$ e 
$\hat{\mathbf{u}}(\mathbf{q}) \equiv \hat{\mathbf{u}}(I, D)$ são vetores unitários
definidos pela equação \ref{eq:u-hat} em função da inclinação e declinação do 
campo principal ($I_{0}$ e $D_{0}$) e da camada equivalente ($I$ e $D$),
respectivamente, e $\mathbf{M}_{ij}$ é uma matriz $3 \times 3$ dada por 
\begin{equation}
\mathbf{M}_{ij} = \begin{bmatrix}
\partial_{xx} \frac{1}{r''} & 
\partial_{xy} \frac{1}{r''} &
\partial_{xz} \frac{1}{r''} \\
\partial_{xy} \frac{1}{r''} & 
\partial_{yy} \frac{1}{r''} &
\partial_{yz} \frac{1}{r''} \\
\partial_{xz} \frac{1}{r''} & 
\partial_{yz} \frac{1}{r''} &
\partial_{zz} \frac{1}{r''}
\end{bmatrix} \quad ,
\label{eq:Mij-matrix}
\end{equation}
em que $\partial_{\alpha\beta} \frac{1}{r''} \equiv \frac{\partial^{2}}{\partial \alpha \partial \beta} \frac{1}{r''}$ 
representa a segunda derivada da função $\frac{1}{r''}$ (equação \ref{eq:inv-r''}) em relação a $\alpha$ e $\beta$, 
$\alpha = x, y, z$ e $\beta = x, y, z$, avaliada nas coordenadas $(x, y, z) = (x_{i}, y_{i}, z_{i})$ do $i$-ésimo dado  
observado e $(x'', y'', z_{c}) = (x_{j}, y_{j}, z_{c})$ da $j$-ésima fonte equivalente.
A anomalia de campo total predita $\Delta T_{i}(\mathbf{s})$ (equação \ref{eq:tfa-pred-i}) é 
obtida substituindo-se, na equação \ref{eq:Delta-T-layer}, a 
discretização da integral que define a componente $\tilde{B}_{\alpha}(x, y, z)$
(equação \ref{eq:B-alpha-layer}) produzida pela camada equivalente contínua.
As equações $\ref{eq:tfa-pred-i}$-$\ref{eq:Mij-matrix}$ mostram que a anomalia de campo total predita 
$\Delta T_{i}(\mathbf{s})$ possui uma relação linear com o vetor de momentos magnéticos $\mathbf{p}$ e uma 
relação não linear com o vetor de direção de magnetização $\mathbf{q}$ (equação \ref{eq:q-vector}).

%% Figura 

\begin{figure}[H]
	\centering
	\includegraphics[width=.7\textwidth]{Fig/eqlayer/eqlayer_figure_tfa.png}
	\caption{Representação esquemática da camada equivalente para a anomalia de campo total. A camada é posicionada 
	sobre o plano horizontal a uma profundidade $z=z_{c}$. $\Delta T_{i}(\mathbf{s})$ é a anomalia de campo total predita 
	(equação \ref{eq:tfa-pred-i}) no ponto 
	$(x_{i},y_{i},z_{i})$ produzida pelo conjunto de $M$ fontes equivalentes (pontos pretos). Cada fonte é localizada 
	no ponto  $(x_{j},y_{j},z_{c})$, $j = 1, \dots, M$, e são representadas por um dipolo de volume unitário 
	$\upsilon_{j}$ com direção de magnetização $\hat{\mathbf{u}}(\mathbf{q})$ e momento magnético $p_{j}$. 
	$\hat{\mathbf{u}}_{0}$ é um vetor unitário na direção do campo principal.}
	\label{fig:eqlayer_tfa_sketch}
\end{figure}

\subsection{Problema inverso}
\label{subsec:mag_dir_prob_inv}

%%%%% Defining the objective function
Seja $\mathbf{\Delta T}^{o}$ o vetor de dados observados cujo $i$-ésimo elemento $\Delta T_{i}^{o}$ é a anomalia de campo total produzida por uma fonte magnética no ponto $(x_{i},y_{i},z_{i})$, $i = 1, \dots, N$. Similarmente, seja $\mathbf{\Delta T} (\mathbf{s})$ o vetor de dados preditos cujo $i$-ésimo elemento $\Delta T_{i}(\mathbf{s})$ (equação \ref{eq:tfa_pred_i}) é a anomalia de campo total produzida por uma camada equivalente discreta no mesmo ponto $(x_{i},y_{i},z_{i})$. Com o objetivo de estimarmos um vetor de parâmetros $\mathbf{s}$ (equação \ref{eq:parameter-vector}) que minimiza a diferença entre $\mathbf{\Delta T}^{o}$ e $\mathbf{\Delta T}(\mathbf{s})$, temos que resolver o problema inverso:

\begin{subequations}
	\begin{align}
	& \text{minimizar}
	& &\Psi(\mathbf{s}) =\lVert \mathbf{\Delta T}^{o} - \mathbf{\Delta T} (\mathbf{s}) 
	\rVert_{2}^{2} + \, \mu f_0 \parallel \mathbf{p} \parallel_{2}^{2} \: , \\
	& \text{sujeito a}
	& & \mathbf{p} \geqslant \mathbf{0} \: .
	\end{align}
	\label{eq:positivity_goal_function}
\end{subequations}
O primeiro e o segundo termo da equação \ref{eq:positivity_goal_function}a são, respectivamente, a função de ajuste e a função 
regularizadora de Tikhonov de ordem zero, $\mu$ é o parâmetro de regularização, $\| \cdot \|_{2}^{2}$ representa o quadrado da 
norma Euclidiana e $f_{0}$ é um fator de normalização. Na inequação \ref{eq:positivity_goal_function}b, $\mathbf{0}$ é um vetor 
$M \times 1$ com todos os elementos iguais a zero no qual o sinal da inequação é aplicado elemento a elemento. 
Este vínculo de positividade sobre o vetor de momentos magnéticos $\mathbf{p}$ é incorporado utilizando o chamado 
\textit{estimador de mínimos quadrados não negativo} ou somente NNLS (do inglês \textit{Nonnegative least squares}) proposto 
por \cite{lawson_hanson_1974}.

Para resolver este problema inverso, temos que considerar primeiramente uma expansão até segunda ordem da função objetivo 
(equação \ref{eq:positivity_goal_function}a) em torno de $\mathbf{s} = \mathbf{s}^{k}$ (equação \ref{eq:parameter-vector}):

\begin{equation}
\Psi(\mathbf{s}^{k} + \mathbf{\Delta s}^{k}) \approx \Psi(\mathbf{s}^{k}) + 
{\mathbf{J}^{k}}^{\top} \mathbf{\Delta s}^{k} + 
\frac{1}{2} {\mathbf{\Delta s}^{k}}^{\top} \mathbf{H}^{k} \mathbf{\Delta s}^{k}  \: ,
\label{eq:sec_ord_goal}
\end{equation}
em que $\mathbf{\Delta s}^{k}$ é uma perturbação no vetor de parâmetros e os termos $\mathbf{J}^{k}$ e $\mathbf{H}^{k}$ são, respectivamente, o vetor gradiente e a matriz Hessiana avaliadas em $\mathbf{s}^{k}$. Então, estimamos o vetor de perturbação $\bar{\mathbf{\Delta s}}^k$ que minimiza a função expandida (equação \ref{eq:sec_ord_goal}) tomando o seu gradiente e igualando o resultado ao vetor nulo. Este procedimento nos leva ao sistema linear 

\begin{equation}
\mathbf{H}^{k} \bar{\mathbf{\Delta s}}^{k} = - \mathbf{J}^{k} \: ,
\label{eq:linear_sys_GN}
\end{equation}
que representa o $k$-ésimo passo do método de Gauss-Newton \citep{aster2005} para a minimização da função objetivo (equação \ref{eq:positivity_goal_function}a). Reescrevemos este sistema linear desprezando as derivadas cruzadas na matriz Hessiana como: 

\begin{equation}
\left[
\begin{array}{c|c}
\mathbf{H}_{pp}^{k} & \mathbf{0} \\
\hline
\mathbf{0}^{\top} & \mathbf{H}_{qq}^{k}
\end{array}
\right] \left[ \begin{array}{c}
\bar{\mathbf{\Delta p}}^{k} \\ 
\bar{\mathbf{\Delta q}}^{k} 
\end{array} \right] \approx -\left[ \begin{array}{c}
\mathbf{J}_{p}^{k} \\ 
\mathbf{J}_{q}^{k} 
\end{array} \right] ,
\label{eq:linear_sys_GN_block}
\end{equation}
em que $\mathbf{0}$ é uma matriz $M \times 2$ que contém todos os elementos iguais a zero, $\bar{\mathbf{\Delta p}}^{k} = \bar{\mathbf{p}}^{k+1} - \bar{\mathbf{p}}^{k}$ é a correção no vetor de momentos magnéticos $\mathbf{p}$, $\bar{\mathbf{\Delta q}}^{k} = \bar{\mathbf{q}}^{k+1} - \bar{\mathbf{q}}^{k}$ é a correção no vetor direção de magnetização e os termos $\mathbf{J}_{\alpha}^{k}$ e $\mathbf{H}_{\alpha \alpha}^{k}$, $\alpha = p,q$, são os vetores gradientes e as matrizes Hessianas calculadas com respeito aos elementos dos $\mathbf{p}$ e $\mathbf{q}$, respectivamente. 

O vetor gradiente $\mathbf{J}_{p}^{k}$ e a matriz Hessiana $\mathbf{H}_{pp}^{k}$ (equação \ref{eq:linear_sys_GN_block}) relativas ao vetor de momentos magnéticos $\mathbf{p}$ (equação \ref{eq:parameter-vector}) são, respectivamente, 

\begin{equation}
\mathbf{J}_{p}^{k} = -2 {\mathbf{G}_{p}^{k}}^{\top} 
\left[ \mathbf{\Delta T}^{o} - \mathbf{\Delta T} (\bar{\mathbf{s}}^{k}) \right] + 
2\mu f_{0}^{k} \bar{\mathbf{p}}^{k} 
\label{eq:grad_p}
\end{equation}
e 

\begin{equation}
\mathbf{H}_{pp}^{k} = 2 {\mathbf{G}_{p}^{k}}^{\top} \mathbf{G}_{p}^{k} + 
2 \mu f_{0}^{k} \mathbf{I} \: ,
\label{eq:hess_p}
\end{equation}
em que $\mathbf{G}_p^{k}$ é uma matriz de dimensão $N \times M$ cujo $ij$-ésimo elemento é dado pela função harmônica $g_{ij}(\bar{\mathbf{q}}^{k})$ (equação \ref{eq:g_ij}) avaliada na direção de magnetização $\bar{\mathbf{q}}^{k}$, $\mathbf{I}$ é uma matriz identidade de dimensão $M \times M$ e $f_{0}^{k}$ é um fator de normalização igual a 

\begin{equation}
f_{0}^{k} = \dfrac{trace \left({\mathbf{G}_{p}^{k}}^{\top} \mathbf{G}_{p}^{k} \right)}{M} \, .
\label{eq:norm_factor}
\end{equation}

O vetor gradiente $\mathbf{J}_{q}^{k}$ e a matriz Hessiana $\mathbf{H}_{qq}^{k}$ (equação \ref{eq:linear_sys_GN_block}) relativas a direção de magnetização $\mathbf{q}$ (equação \ref{eq:q_vetor}) são, respectivamente, 

\begin{equation}
\mathbf{J}_{q}^{k} = -2 {\mathbf{G}_{q}^{k}}^{\top} 
\left[ \mathbf{\Delta T}^{o} - \mathbf{\Delta T} (\bar{\mathbf{s}}^{k}) \right]
\label{eq:grad_q}
\end{equation}
e

\begin{equation}
\mathbf{H}_{qq}^{k} \approx 2 {\mathbf{G}_{q}^{k}}{^\top} \mathbf{G}_{q}^{k} \: ,
\label{eq:hess_q}
\end{equation}
em que $\mathbf{G}_{q}^{k}$ é uma matriz $N \times 2$ dada por 

\begin{equation}
\mathbf{G}_{q}^{k} = \begin{bmatrix}
\partial_{I} \mathbf{g}_{1}(\bar{\mathbf{q}}^{k})^{\top} \bar{\mathbf{p}}^{k} & 
\partial_{D} \mathbf{g}_{1}(\bar{\mathbf{q}}^{k})^{\top} \bar{\mathbf{p}}^{k} \\
\vdots & \vdots  \\
\partial_{I} \mathbf{g}_{N}(\bar{\mathbf{q}}^{k})^{\top} \bar{\mathbf{p}}^{k} & 
\partial_{D} \mathbf{g}_{N}(\bar{\mathbf{q}}^{k})^{\top} \bar{\mathbf{p}}^{k} 
\end{bmatrix} \: ,
\label{eq:Gq}
\end{equation}
em que $\partial_{\alpha} \mathbf{g}_{i}(\bar{\mathbf{q}}^{k}) \equiv \frac{\partial \mathbf{g}_{i}(\bar{\mathbf{q}}^{k})}{\partial \alpha}$, $\alpha= I, D$, representa a primeira derivada do vetor $\mathbf{g}_{i}(\bar{\mathbf{q}}^{k})$ (equação \ref{eq:tfa_pred_i}) com respeito a inclinação $I$ e a declinação $D$ da magnetização total das fontes.

\subsection{Processo iterativo para a estimativa da direção de magnetização}

A iteração $k=0$ do nosso algoritmo começa com uma aproximação inicial $\bar{\mathbf{q}}^{k} = \bar{\mathbf{q}}^{0}$ para o vetor direção de magnetização $\mathbf{q}$ (equação \ref{eq:q_vetor}). Utilizando esta aproximação inicial $\bar{\mathbf{q}}^{k}$, a parte superior da equação \ref{eq:linear_sys_GN_block} nos leva ao seguinte sistema linear para o vetor de momentos magnéticos:

\begin{equation}
\left[ {\mathbf{G}_{p}^{k}}^{\top} \mathbf{G}_{p}^{k} + 
\mu f_{0}^{k} \mathbf{I} \right] \bar{\mathbf{p}}^{k} = {\mathbf{G}_{p}^{k}}^{\top} \mathbf{\Delta T}^{o} \: .
\label{eq:linear_sys_p}
\end{equation}
Para impor o vínculo de positividade (equação \ref{eq:positivity_goal_function}b) sobre a distribuição de momentos magnéticos, este sistema linear é resolvido usando o método de NNLS \citep{lawson_hanson_1974, silvadias_etal_2010}. Esta distribuição de momentos magnéticos é então usada para estimar uma correção $\bar{\mathbf{\Delta q}}^{k}$ no vetor direção de magnetização resolvendo um sistema não linear utilizando o método de Levenberg-Marquardt \citep{aster2005}:

\begin{equation}
\left[ {\mathbf{G}_{q}^{k}}^{\top} \mathbf{G}_{q}^{k} + \lambda \, \mathbf{I} \right] 
\bar{\mathbf{\Delta q}}^{k} = {\mathbf{G}_{q}^{k}}^{\top} 
\left[ \mathbf{\Delta T}^{o} - \mathbf{\Delta T} (\mathbf{s}^{k}) \right] \: ,
\label{eq:linear_sys_q}
\end{equation}
em que $\lambda$ é o parâmetro de Marquardt e $\mathbf{I}$ é uma matriz identidade. Após estimarmos a correção $\bar{\mathbf{\Delta q}}^{k}$ na $k$-ésima iteração, atualizamos a direção de magnetização aplicando a correção a seguir:

\begin{equation}
\bar{\mathbf{q}}^{k+1} = \bar{\mathbf{q}}^{k} + \bar{\mathbf{\Delta q}}^{k} \: ,
\label{eq:q_next}
\end{equation}
e utilizando esta nova direção para estimar uma nova distribuição de momentos magnéticos com a equação \ref{eq:linear_sys_p} e assim sucessivamente. O processo iterativo é interrompido quando a função objetivo (equação \ref{eq:positivity_goal_function}a) é invariante ao longo de sucessivas iterações. Mostramos também que este método falha em situações nas quais as fontes são magnetizadas verticalmente (Apêndice \ref{append:vertical-magnetization}).

\subsection{Limitação para o caso de fontes magnetizadas verticalmente}
\label{subsec:vertical-magnetization}

Nosso método falha quando a magnetização total das fontes possui a direção igual ou aproximadamente vertical. Neste apêndice, fornecemos a base teórica para o entendimento desta limitação. 

Considere o caso limite no qual a magnetização total das fontes é vertical e.g., $I = \pm 90^\circ$). Neste caso, a anomalia de campo total $\Delta T(x, y, z)$ (equação \ref{eq:tfanomaly}) não depende da declinação $D$, demonstrado pelo fato que: fontes magnetizadas verticalmente não possuem uma declinação definida. Consequentemente, o mínimo da função objetivo (equação \ref{eq:positivity_goal_function}a) não é bem definida no espaço dos parâmetros; isto é, ela é alongada na direção de $D$. Infelizmente, o vínculo de positividade sobre o vetor de momentos magnéticos (equação  \ref{eq:positivity_goal_function}b) não resolve esta ambiguidade com respeito a declinação $D$. 

Para melhor entender como esta ambiguidade afeta nosso método, começamos a analisar a matriz $\mathbf{G}_{q}^{k}$ de dimensão $N \times 2$ (equação \ref{eq:Gq}) necessária para estimar a correção $\bar{\mathbf{\Delta q}}^{k}$ para a direção de magnetização (equação \ref{eq:linear_sys_q}). Sua $i$-ésima linha é definida como o produto do vetor de momentos magnéticos estimado $\bar{\mathbf{p}}^{k}$ e as primeiras derivadas $\partial_{\alpha} \mathbf{g}_{i}(\bar{\mathbf{q}}^{k}) \equiv 
\frac{\partial \mathbf{g}_{i}(\bar{\mathbf{q}}^{k})}{\partial \alpha}$, $\alpha= I, D$, do vetor $\mathbf{g}_{i}(\mathbf{q})$ (equação \ref{eq:tfa_pred_i}), avaliada em $\mathbf{q} = \bar{\mathbf{q}}^{k}$, com respeito a inclinação $I$ e a declinação $D$ da magnetização total das fontes. O $j$-ésimo elemento $\partial_{\alpha} g_{ij}(\bar{\mathbf{q}}^{k}) \equiv 
\frac{\partial g_{ij}(\bar{\mathbf{q}}^{k})}{\partial \alpha}$ do vetor $\partial_{\alpha} \mathbf{g}_{i}(\bar{\mathbf{q}}^{k})$ de dimensão $M \times 1$ é definido como a derivada da função harmônica $g_{ij}(\mathbf{q})$ (equação \ref{eq:g_ij}) igual a 

\begin{equation}
\partial_{\alpha} g_{ij}(\bar{\mathbf{q}}^{k}) = 
\gamma_m  \hat{\mathbf{F}}_{0}^T \, \mathbf{M}_{ij} 
\partial_{\alpha} \hat{\mathbf{m}}(\bar{\mathbf{q}}^{k}) \: , \quad \alpha = I, D \: ,
\label{eq:D-alpha-gij}
\end{equation}
em que 

\begin{equation}
\partial_{I} \hat{\mathbf{m}}(\bar{\mathbf{q}}^{k}) = 
\begin{bmatrix}
	-\sin \bar{I}^{k} \cos \bar{D}^{k} \\
	-\sin \bar{I}^{k} \sin \bar{D}^{k} \\
	 \cos \bar{I}^{k}
\end{bmatrix}
\label{eq:D_mag_vec_inc}
\end{equation}
e 

\begin{equation}
\partial_{D} \hat{\mathbf{m}}(\bar{\mathbf{q}}^{k}) = 
\begin{bmatrix}
	-\cos \bar{I}^{k} \sin \bar{D}^{k} \\
	 \cos \bar{I}^{k} \cos \bar{D}^{k} \\
	 0
\end{bmatrix}
\label{eq:D_mag_vec_dec}
\end{equation}
são as derivadas do vetor unitário $\hat{\mathbf{m}}(\mathbf{q})$ (equação \ref{eq:mag_vet}), avaliadas na direção de magnetização $\bar{\mathbf{q}}^{k} = \left[ \bar{I}^{k} \:\: \bar{D}^{k} \right]^{\top}$, com respeito a $I$ e $D$. 
 
Note que, quando a inclinação estimada $\bar{I}^{k}$ se aproxima de $\pm 90^{\circ}$, todos os elementos que formam o vetor  $\partial_{D} \hat{\mathbf{m}}(\bar{\mathbf{q}}^{k})$ (equação \ref{eq:D_mag_vec_dec})e, consequentemente, a segunda coluna da matriz $\mathbf{G}_{q}^{k}$ (equação \ref{eq:Gq}) tendem a zero. Como consequência, o problema não-linear para estimar a direção de magnetização (equação \ref{eq:linear_sys_q}) não é sensível a mudanças na declinação $D$ e a convergência do nosso método é muito lenta devido a suavidade da função objetivo $\Psi(\mathbf{s})$ (equação \ref{eq:positivity_goal_function}a) no espaço de parâmetros. 


\section{Profundidade da camada ($\mathbf{z_{c}}$) e parâmetro de regularização ($\mathbf{\mu}$)}

O procedimento pelo qual utilizamos a camada equivalente para estimar a direção de magnetização total das fontes magnéticas e o cálculo das componentes do campo magnético requer a escolha de dois parâmetros principais. O primeiro é a profundidade da camada $z_c$ (Figuras \ref{fig:eqlayer_tfa_sketch} e \ref{fig:eqlayer_bz_sketch}) e o segundo é o parâmetro de regularização $\mu$ mostrado na equação \ref{eq:linear_sys_p}. 

O método utilizado para a escolha da profundidade da camada é baseado na abordagem clássica proposta por \cite{dampney1969}. O autor aponta que o posicionamento da camada deve satisfazer um intervalo de $2,5$ a $6,0$ vezes o espaçamento dos dados. Vale ressaltar que esta regra foi aplicada pelos autores em uma grade com dados regularmente espaçados. Contudo, a escolha para aplicar nosso método corresponde a um intervalo de $2$ a $3$ vezes o valor do maior espaçamento entre os dados. É necessário lembrar que este intervalo foi encontrado empiricamente. 

Para resolver a equação \ref{eq:linear_sys_p}, temos que escolher um valor confiável para o parâmetro de regularização. Com este propósito, usamos o método da curva-L, que serve como uma filtragem de ruídos dos dados, sem que o resultado final perca informações. O 'cotovelo' desta curva é o valor ótimo de parâmetro no qual é feito o balanço entre a função de ajuste e a função regularizadora. 
%\chapter{Magnetic microscopy data application for Vredefort sample}

The observed data were measured on a regular grid of $121 \times 99$ (a total number of $N = 11979$ observations) over an area extending $\sim 36 \, mm$ and $\sim 30 \, mm$ along the x- and y-axis, respectively (Figure \ref*{fig:data_fitting}a). The sensor-to-sample distance was $138$ microns above the sample surface. We use a layer composed by a grid of $121 \times 99$ dipoles (a total of $M=11979$ equivalent sources) positioned at a constant depth of $z=750$ microns. The magnetization direction for all dipoles composing the equivalent layer is equal to $90^\circ$ and $0^\circ$ for inclination and declination, respectively. That is, its the same direction of the imparted field. 

By solving the equation \ref*{eq:ls_estimator}, using a regularizing parameter $\mu = 10^{-19}$, we estimate a magnetic-moment distribution over the layer (not shown). Figure \ref*{fig:data_fitting}b is the predicted data produced by equivalent layer. Figure \ref*{fig:data_fitting}c shows the residuals map that is the difference between observed data (Figure \ref*{fig:data_fitting}a) and the pŕedicted data (Figure \ref*{fig:data_fitting}b). The histogram of residuals appears with mean $0 \, mT$ and standard deviation $0,002 \, mT$. It means that the estimate magnetic-moment distribution produces an acceptable data fitting. Figure \ref*{fig:total_field_components}a, \ref*{fig:total_field_components}b and \ref*{fig:total_field_components}c show the predicted z-, x- and y-components of the magnetic field, respectively. We calculate the amplitude of total field using the estimated three components (Figure \ref*{fig:total_field_components}d). As we can notice from the amplitude of total field, there is a concentration of magnetic carriers on the upper bound of the Vredefort sample.      


%%%%%%%%%% Figures %%%%%%%%%%%%%
\begin{figure}[h!]
\centering
\includegraphics[width=0.8\textwidth]{vred/figs/results_data_fitting_eqlayer.png}
\caption{Application to microscopy data from Vredefort sample. (a) Observed z-component measured by magnetic microscope. (b) Estimated z-component produced by the model. (c) Difference between panels a and b. (d) Histogram of residuals. }
\label{fig:data_fitting}
\end{figure}

\begin{figure}[h!]
\centering
\includegraphics[width=0.6\textwidth]{vred/figs/field_components_eqlayer.png}
\caption{Application to microscopy data from Vredefort sample. (a) Map of z-component produced by the model. (b)  Map of estimated x-component. (c) Map of estimated y-component. (d) Total field calculated from the estimated magnetic-field components.}
\label{fig:total_field_components}
\end{figure}


 


 %\include{vred/conclusao}

%\appendix

\backmatter

% estilo de citações por ordem alfabética (defaut da classe ONTeX)
%\bibliographystyle{on-plain}
%\bibliography{references.bib}

  
\end{document} 