\append{Deduction of equation \ref{eq:positivity_prop}}
\label{append:proof-positive-p}

In this appendix, we prove the existence of an all-positive magnetic-moment 
distribution $p(x'', y'', z_{c})$ that solve the integral shown in equation 
\ref{eq:Gamma_integral_equation}.

Consider a closed surface located above the magnetic sources, formed by the plane $z = z_{c}$ containing the equivalent layer and a hemisphere with infinite radius
(Figure \ref{fig:surface_Green}).
This surface encloses a region where $\Gamma(x'', y'', z_{c})$ 
(equation \ref{eq:Gamma-volume-integral}) is harmonic at all points.
By using the Green's second identity \citep[][ p. 215]{kellogg1967}, we can show that
\begin{equation}
0 = \frac{1}{4\pi}
\int\limits_{-\infty}^{+\infty}\int\limits_{-\infty}^{+\infty}
\partial_{z} \Gamma(x'', y'', z_{c}) \: \frac{1}{\ell} - 
\Gamma(x'', y'', z_{c}) \: \partial_{z} \frac{1}{\ell}
\:\: dS'' \: , \quad z_{c} > z \: ,
\label{eq:Greens_2nd_identity}
\end{equation}
where $\Gamma(x'', y'', z_{c})$ is the volume integral defined by equation 
\ref{eq:Gamma-volume-integral} and
\begin{equation}
\frac{1}{\ell} \equiv \frac{1}{\sqrt{(x - x'')^{2} +
		(y - y'')^{2} +
		(z_{s} - z_{c})^{2}}}
\label{eq:inv-l}
\end{equation}
is the inverse distance between the fixed point $(x'', y'', z_{c})$, located on the
equivalent layer, and the point $(x, y, z_{s})$, with $z_{s} = z_{c} + \Delta z$,
$\Delta z > 0$.
The point $(x, y, z_{s})$ is conveniently defined as the mirror of $(x, y, z)$,
located at $z = z_{c} - \Delta z$, with respect to the plane $z = z_{c}$ containing
the equivalent layer (Figure \ref{fig:surface_Green}).
Equation \ref{eq:Greens_2nd_identity} combined with the 
Green's third identity \citep[][ p. 219]{kellogg1967} gives rise to
\begin{equation}
\Gamma(x, y, z) = \frac{1}{4\pi}
\int\limits_{-\infty}^{+\infty}\int\limits_{-\infty}^{+\infty}
\partial_{z} \Gamma(x'', y'', z_{c}) \: 
\left( \frac{1}{r} + \frac{1}{\ell} \right)
\Gamma(x'', y'', z_{c}) \: 
\left( \partial_{z} \frac{1}{r} + \partial_{z} \frac{1}{\ell} \right)
\:\: dS'' \: , \quad z_{c} > z \: ,
\label{eq:Greens_3rd_identity}
\end{equation}
where $\frac{1}{r}$ is defined by equation \ref{eq:inverse-distance}.
The term $\left( \frac{1}{r} + \frac{1}{\ell} \right)$ represents the 
\textit{Green's function of second kind} \citep[][ p. 246]{kellogg1967} 
associated with this integral.
We can verify that $\frac{1}{r} = \frac{1}{\ell}$, $\partial_{z} (1/r) = -\partial_{z} (1/\ell)$
and, consequently, 
\begin{equation}
\Gamma(x, y, z) = \frac{1}{2\pi}
\int\limits_{-\infty}^{+\infty}\int\limits_{-\infty}^{+\infty}
\partial_{z} \Gamma(x'', y'', z_{c}) \: \frac{1}{r} 
\:\: dS'' \: , \quad z_{c} > z \: .
\label{eq:Neumann_bvp}
\end{equation}
This equation shows the inherent ambiguity of potential field methods
\citep{roy1962} and solves the \textit{Neumann's problem} or the 
\textit{second boundary value problem of potential theory} \citep[][ p. 246]{kellogg1967}.
In the present case, this problem consists in defining the harmonic function $\Gamma(x, y, z)$ 
(equation \ref{eq:Gamma-volume-integral}) at the region above the equivalent layer,
from the values of its vertical derivative on the plane containing the equivalent layer. 