\begin{abstract}
It is known from potential theory that a continuous and planar layer of dipoles can exactly reproduce the total-field anomaly produced by arbitrary 3D sources. We show that, if the magnetization direction of this layer is the same as that of the true sources, its magnetic-moment distribution is all positive. This property holds true regardless of whether the magnetization of the true sources is purely induced or not. By using this generalized positivity constraint, we present a new iterative method for estimating the total magnetization direction of 3D magnetic sources based on the equivalent-layer technique. Our method does not impose a priori information either about the shape or depth of the sources, does not require regularly spaced data and presumes that the sources have a uniform magnetization direction. At each iteration, our method performs two steps. The first one solves a constrained linear inverse problem to estimate a positive magnetic-moment distribution over a discrete equivalent layer of dipoles. We consider that the equivalent sources are located on a plane and have an uniform and fixed magnetization direction. The second step uses the estimated magnetic-moment distribution and solves a nonlinear inverse problem for estimating a new magnetization direction for the dipoles. The algorithm stops when the equivalent layer yields a total-field anomaly that fits the observed data. Tests with synthetic data simulating different geological scenarios show that the final estimated magnetization direction is close to the true one. We applied our method to a field data from the Goi{\' a}s Alkaline Province (GAP), over the Montes Claros complex, center of Brazil. The result suggests the presence of intrusions with remarkable remanent magnetization, in agreement with the current literature for this region.
\end{abstract}