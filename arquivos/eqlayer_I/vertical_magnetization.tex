\append{Vertically magnetized sources}
\label{append:vertical-magnetization}

Our method fails when the total-magnetization of the sources 
has a direction equal or close to vertical. In this appendix, we provide the theoretical basis for 
understanding this limitation.

Consider the limiting case in which the total-magnetization of the sources is vertical 
(e.g., $I = \pm 90^\circ$). In this case, the total-field anomaly $\Delta T(x, y, z)$ 
(equation \ref{eq:tfanomaly}) does not depend on the declination $D$, which shows a 
well-known fact: vertically magnetized sources do not have a definite declination.
As a consequence, the minimum region of the goal function (equation 23a) on the parameter 
space is not well defined; rather, it is  elongated in the direction of D. 
Unfortunately, the positivity constraint on the magnetic-moment
vector (equation \ref{eq:positivity_goal_function}b) does not solve this ambiguity with 
respect to the declination $D$.

To understand how this ambiguity affects our method, let us start by analyzing the 
$N \times 2$ matrix $\mathbf{G}_{q}^{k}$ (equation \ref{eq:Gq}) required for 
estimating the correction $\bar{\mathbf{\Delta q}}^{k}$ in the magnetization direction
(equation \ref{eq:linear_sys_q}). 
Its $i$th line is defined by the dot product of the estimated magnetic-moment vector 
$\bar{\mathbf{p}}^{k}$ and the first derivatives 
$\partial_{\alpha} \mathbf{g}_{i}(\bar{\mathbf{q}}^{k}) \equiv 
\frac{\partial \mathbf{g}_{i}(\bar{\mathbf{q}}^{k})}{\partial \alpha}$, $\alpha= I, D$, 
of the vector $\mathbf{g}_{i}(\mathbf{q})$ (equation \ref{eq:tfa_pred_i}),
evaluated at $\mathbf{q} = \bar{\mathbf{q}}^{k}$, with respect to the inclination 
$I$ and the declination $D$ of the total magnetization of the sources.
The $j$th element 
$\partial_{\alpha} g_{ij}(\bar{\mathbf{q}}^{k}) \equiv 
\frac{\partial g_{ij}(\bar{\mathbf{q}}^{k})}{\partial \alpha}$ 
of the $M \times 1$ vector $\partial_{\alpha} \mathbf{g}_{i}(\bar{\mathbf{q}}^{k})$ is defined
by computing the derivative of the harmonic function $g_{ij}(\mathbf{q})$ (equation \ref{eq:g_ij}), 
as follows:
\begin{equation}
\partial_{\alpha} g_{ij}(\bar{\mathbf{q}}^{k}) = 
\gamma_m  \hat{\mathbf{F}}_{0}^T \, \mathbf{M}_{ij} 
\partial_{\alpha} \hat{\mathbf{m}}(\bar{\mathbf{q}}^{k}) \: , \quad \alpha = I, D \: ,
\label{eq:D-alpha-gij}
\end{equation}
where
\begin{equation}
\partial_{I} \hat{\mathbf{m}}(\bar{\mathbf{q}}^{k}) = 
\begin{bmatrix}
	-\sin \bar{I}^{k} \cos \bar{D}^{k} \\
	-\sin \bar{I}^{k} \sin \bar{D}^{k} \\
	 \cos \bar{I}^{k}
\end{bmatrix}
\label{eq:D_mag_vec_inc}
\end{equation}
and
\begin{equation}
\partial_{D} \hat{\mathbf{m}}(\bar{\mathbf{q}}^{k}) = 
\begin{bmatrix}
	-\cos \bar{I}^{k} \sin \bar{D}^{k} \\
	 \cos \bar{I}^{k} \cos \bar{D}^{k} \\
	 0
\end{bmatrix}
\label{eq:D_mag_vec_dec}
\end{equation}
are derivatives of the unit vector $\hat{\mathbf{m}}(\mathbf{q})$ (equation \ref{eq:mag_vec}), 
evaluated at the magnetization direction 
$\bar{\mathbf{q}}^{k} = \left[ \bar{I}^{k} \:\: \bar{D}^{k} \right]^{\top}$, 
with respect to $I$ and $D$.

Notice that, as the estimated inclination $\bar{I}^{k}$ approaches $\pm 90^{\circ}$, 
all elements forming the vector $\partial_{D} \hat{\mathbf{m}}(\bar{\mathbf{q}}^{k})$ 
(equation \ref{eq:D_mag_vec_dec}) and, consequently, those forming the second column of 
$\mathbf{G}_{q}^{k}$ (equation \ref{eq:Gq}) tend to zero. 
As a consequence, the nonlinear problem for estimating the magnetization direction 
(equation \ref{eq:linear_sys_q}) becomes insensitive to changes in the declination $D$ 
and the convergence of our method becomes very slow due to the flatness of the goal function 
$\Psi(\mathbf{s})$ (equation \ref{eq:positivity_goal_function}a) in the parameter space.

