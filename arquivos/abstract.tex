\begin{foreignabstract}

In this thesis, I present two theoretical results and the applications of the 
equivalent layer for processing and interpreting magnetic data.
In the first one, I show that there is a unique planar and continuous layer of dipoles,
with a given uniform magnetization direction, that is able to reproduce, simultaneously,
the three components of the magnetic induction field produced by an arbitrary set of 
sources. This property holds true regardless of whether the magnetization direction of 
the layer is equal to the that of the sources or not.
From this theoretical result, I show that it is possible to use a planar layer of dipoles 
with uniform and arbitrary magnetization direction to estimate the three components of the 
magnetic induction field produced by an arbitrary set of sources via linear inversion
of single component data. 
Results with synthetic data produced by numerical simulations and real data obtained 
on a rock sample from the Vredefort impact crater, South Africa, show the utility of the method
in the processing of magnetic microscopy data and identification of regions with largest 
concentrations of magnetic minerals.
In the second theoretical development presented in this thesis, I show that the magnetic 
moment intensity distribution on a planar and continuous layer of dipoles is all positive 
if the uniform magnetization direction of the layer is equal to that of the true sources.
Using this positivity property, I present an iterative method for estimating the uniform 
magnetization direction of a set of 3D sources by inverting total-field anomaly data.
At each iteration, the method solves a linear inverse problem to estimate a positive magnetic 
moment intensity distribution and a non-linear inverse problem to estimate the magnetization 
direction on a planar layer of dipoles. At the end of the iterative process, the uniform 
magnetization direction of the equivalent sources approximates that of the true sources.
Tests with data produced by models simulating different geological scenarios show that the method 
can be a powerful tool for estimating the uniform magnetization direction of a set of geological 
sources. Applications to airborne data over the Montes Claros de Goiás complex, located in the 
Goiás Alkaline Province, central region of Brazil, suggest that those intrusions have a strong 
remanent magnetization, in agreement with a previous independent study in the same area.

\end{foreignabstract}