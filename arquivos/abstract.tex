\begin{foreignabstract}
The equivalent-layer technique is commonly used for processing total-field anomaly data by estimating a 2D magnetic-moment distribution over a fictitional layer composed by dipoles below the observation plane. However, some processing approaches using equivalent layer technique depends on the knowledge of the magnetization direction of sources. In this work, we investigate two interesting theoretical and practical aspects of this technique. The first one, we developed a new method for estimating the total magnetization direction of magnetic sources based on equivalent-layer technique using total-field anomaly data. When the magnetization direction of equivalent sources is almost the same as the true body, the estimated magnetic property over the layer is all positive. Iteratively, the proposed method imposes zeroth-order Tikhonov regularization and positivity constraint on the estimated magnetic moment over the layer and estimate the magnetization direction of the geological sources. Mathematically, the algorithm solves least-squares problems in two steps: the first one solves a linear inverse problem for estimating a 2D magnetic-moment distribution within the equivalent layer and the second solves a nonlinear inverse problem for magnetization direction of the magnetized sources. For the second approach, we test if the application of equivalent-layer technique for calculating the components and the amplitude of the magnetic field generated by geological sample depends on the prior knowledge of magnetization direction. We set a magnetization direction and estimate a magnetic moment distribution that can be used to calculate the components of magnetic field and its amplitude. In both cases, this approach does not impose strong information either about the shape or about the depth of the sources, and does not require a regularly spaced data. We test the proposed methodology by applying to synthetic data for different geological scenarios, and the results show that the method can be a powerful tool for estimating the magnetization direction of a set of bodies. Tests on field data from Goias Alkaline Province (GAP), center of Brazil, over Montes Claros complex suggests intrusions with remarkable strong remanent magnetization, in agreement with the current literature for this region. Moreover, we test the processing approach with synthetic data simulating rock samples, and the results show that this kind of processing does not depend of the knowledge of magnetization direction. By using magnetic microscopy data, the application of this processing approach in a geological sample from the Vredefort impact crater confirms this feature of the equivalent-layer technique.   
\end{foreignabstract}