\chapter{Conclusões}
\label{chap:conclusion}

Neste trabalho mostramos teoricamente que a anomalia de campo total causada por fontes magnéticas com direção de magnetização uniforme pode ser reproduzida por uma camada plana e contínua com distribuição de momentos magnéticos positiva. Esta propriedade teórica é válida para os casos nos quais a camada equivalente possui a mesma direção de magnetização das fontes verdadeiras, ainda que a magnetização seja puramente induzida ou não. Utilizando este vínculo de positividade, apresentamos um método iterativo para estimar a magnetização total de fontes 3D baseados na técnica da camada equivalente. A cada iteração, aplicamos este vínculo de positividade sobre a distribuição de momentos magnéticos estimada da camada e resolvemos um problema inverso não-linear para estimar a direção de magnetização das fontes equivalente. Este método não requer nenhum conhecimento prévio sobre a forma e a profundidade das fontes magnéticas, nem mesmo o uso de dados regulamente espaçados. Esta abordagem pode ser aplicada para determinar a direção de magnetização de múltiplas fontes, presumindo que total tenham a mesma direção de magnetização. Provamos matematicamente também que, utilizando somente uma das componentes campo magnético, a técnica camada equivalente é capaz de reproduzir as componentes e a amplitude do campo produzido por fontes magnéticas sem o conhecimento prévio de sua direção de magnetização. Neste caso, fixamos uma direção de magnetização arbitrária para a camada equivalente e estimamos uma distribuição de momentos magnéticos resolvendo um problema inverso linear. Através de uma transformação conseguimos calcular as componentes e a amplitude do campo magnético. Esta propriedade é válida para qualquer tipo de magnetização, seja ela induzida ou não, e também para dados que não estão regularmente espaçados. 

Resultados obtidos com dados sintéticos produzidos por múltiplas fontes mostraram que a direção de magnetização pode ser recuperada através do nosso método iterativo. Estes testes também ilustraram como a presença de fontes rasas afetam o resultado obtido por nosso método, para os casos nos quais estas fontes rasas possuem direções de magnetização iguais ou diferentes das demais. Em ambos os casos, a camada equivalente produziu grandes resíduos logo acima das fontes rasas; no entanto, não podemos distinguir se a direção de magnetização da fonte rasa é igual ou diferente das demais. Além disso, nosso método não produz resultados satisfatórios se a fonte rasa tem direção de magnetização diferente das outras fontes. Aplicamos o método a dados de campo provenientes da província alcalina de Goiás, região central do Brasil, confirmaram que nosso método pode ser uma confiável ferramenta para a interpretação de cenários geológicos complexos. Os resultados para as anomalias no complexo de Montes Claros de Goiás sugerem a presenta de uma forte componente remanente na magnetização destas fontes, que corroboram com estudos anteriores conduzidos nesta mesma área. A distribuição de momentos magnéticos estimada sobre a camada nos leva a uma aceitável redução ao polo, mas também produz grandes desajustes em algumas áreas isoladas. Consideramos que estes desajustes locais são devido a fontes rasa e, no entanto, não conseguimos inferir se elas possuem a mesma direção de magnetização das outras fontes. Vale ressaltar também que, caso a fonte esteja magnetizada verticalmente, o método não é capaz de recuperar uma direção de magnetização para estes corpos.  

Além disso, resultados com dados sintéticos produzidos por modelos que simulam amostras de laboratório mostraram que a técnica da camada equivalente pode recuperar as componentes e a amplitude do campo magnético sem termo o conhecimento prévio da direção de magnetização das fontes. Estes testes ilustram que este tipo de processamento pode ser aplicado a dados gerados por microscópios magnéticos. Este fato é confirmado através da aplicação da técnica da camada equivalente no processamento das medidas da componente vertical do campo magnético gerado por uma amostra de rocha proveniente da cratera de Vredefort, na África do Sul. Diferente das técnicas no domínio de Fourier, calculamos as componentes do campo magnético e a sua amplitude no domínio do espaço. Estes resultados mostram que esse tipo de processamento pode ser uma ferramenta confiável para descrever a distribuição de magnetização sobre uma amostra de rocha, identificando regiões livres com ou sem portadores magnéticos ao longo de amostras geológicas.

