\chapter{Conclusões}
\label{chap:conclusion}

Nesta tese apresentamos dois desenvolvimentos teóricos e suas aplicações na interpretação de dados magnéticos 
usando a técnica da camada equivalente. 
No primeiro, provamos matematicamente que, dada uma direção de magnetização uniforme, existe uma única distribuição 
de intensidade de momentos magnéticos na camada equivalente, em uma determinada profundidade constante abaixo das observações, 
que é capaz de reproduzir, simultaneamente, as três componentes de um mesmo campo de indução magnética produzido 
por fontes magnéticas arbitrárias. Uma consequência direta deste resultado teórico é que, se a camada reproduzir  
uma das componentes do campo, ela deve, obrigatoriamente, reproduzir as demais componentes, ainda que sua direção de 
magnetização seja diferente daquela das fontes magnéticas arbitrárias. Caso contrário, as leis de Gauss (para campos magnéticos) 
e de Ampère são violadas. Usando este resultado teórico, fixamos uma direção de magnetização arbitrária para uma camada equivalente 
plana e estimamos uma distribuição de intensidades de momentos magnéticos por meio da inversão linear de dados de uma única componente 
do campo produzido por fontes arbitrárias. Uma vez estimada a distribuição de momento que reproduz uma única componente, 
calculamos as demais componentes e a amplitude do campo de indução magnético. 

Resultados com dados produzidos por modelos que simulam amostras planares de rocha mostraram que o método recupera as componentes 
e a amplitude do campo de indução magnética sem o conhecimento prévio da direção de magnetização das fontes. 
Com base nos resultados com dados sintéticos, aplicamos o método a dados da componente vertical do campo de indução magnética 
gerados por uma amostra de rocha proveniente da cratera de Vredefort, na África do Sul. Os dados foram obtidos por meio de microscopia 
magnética de varredura, usando um sensor de baixo custo baseado no efeito Hall e que opera em temperatura ambiente. 
Os resultados mostram que esse tipo de processamento é muito útil para estimar a distribuição de magnetização em amostras de rocha, 
na escala de laboratório, e identificar regiões com maior concentração de portadores magnéticos. A identificação destas regiões 
é importante para análises posteriores sobre o conteúdo mineralógico e a caracterização magnética das amostras geológicas.

O segundo desenvolvimento teórico apresentado nesta tese mostra que a anomalia de campo total causada por fontes magnéticas com direção 
de magnetização uniforme pode ser reproduzida por uma camada plana e contínua com distribuição de momentos magnéticos positiva. 
Esta propriedade teórica é válida para os casos nos quais a camada equivalente possui a mesma direção de magnetização 
das fontes verdadeiras, seja ela puramente induzida ou não. 
Utilizando esta propriedade de positividade, apresentamos um método iterativo para estimar a magnetização total de fontes 
3D baseada na técnica da camada equivalente. A cada iteração, resolvemos um problema inverso linear para estimar uma distribuição 
de momentos magnéticos positiva e resolvemos um problema inverso não-linear para estimar a direção de magnetização das 
fontes equivalentes. Ao final do processo iterativo, a direção de magnetização na camada se aproxima daquela das 
fontes magnéticas. 
Este método não requer nenhum conhecimento prévio sobre a forma e a profundidade das fontes 
magnéticas, nem mesmo o uso de dados regulamente espaçados. Esta abordagem pode ser aplicada para determinar a direção 
de magnetização de múltiplas fontes, presumindo que todas tenham a mesma direção de magnetização total. 

Resultados obtidos com dados sintéticos produzidos por modelos de múltiplas fontes mostram que a direção de magnetização 
pode ser estimada através do nosso método iterativo. 
Estes testes também ilustram como a presença de fontes muito mais rasas que as demais afeta o resultado obtido por nosso 
método, para os casos em que estas fontes rasas possuem direções de magnetização iguais ou diferentes das demais. 
Em ambos os casos, os resultados mostram grandes resíduos logo acima das fontes rasas; no entanto, não é possível 
distinguir se eles são produzidos por fontes rasas com a mesma direção ou com direção de magnetização diferentes das demais. 
Aplicamos o método a dados de anomalia de campo total provenientes de um aerolevantamento sobre a província alcalina de Goiás, 
região central do Brasil. 
Os resultados confirmaram que nosso método é útil para a interpretação de cenários geológicos complexos. 
Os resultados para as anomalias no complexo de Montes Claros de Goiás sugerem a presenta de magnetização remanente nas fontes e 
estão de acordo com estudos anteriores conduzidos de forma independente na mesma região. 
A distribuição de momentos magnéticos estimada sobre a camada produz uma excelente redução ao polo, com valores predominantemente positivos
e que decaem a zero nas bordas das anomalias. 
Os resultados também mostram grandes desajustes em algumas áreas isoladas. 
Consideramos que estes desajustes locais são devido à presença de fontes rasas. Contudo, não conseguimos inferir se elas possuem a 
mesma direção de magnetização das outras fontes. 


